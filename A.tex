\backgroundsetup
{
    scale=1,
    angle=0,
    opacity=1,
    contents={\begin{tikzpicture}[remember picture,overlay]
        \path [fill=Space] (-\paperwidth, 0.375\paperheight)rectangle (\paperwidth, \paperheight); 
    \end{tikzpicture}}
}
\BgThispage

\begin{A}
\section{Topic A – Space, time and motion}
\subsection{Subtopic A.1 – Kinematics}
\question{A1}{11}{1}
\subquestion{a}
The total distance is given as the sum of distance traveled during each translation, $2.5+3.8=6.3\unit{km}$

\subquestion{b}
Displacement is a vector quantity, thus taking the magnitude of the sum of the two vector displacements caused by the movement:
\begin{align*}
    \magnitude{\vector{2.5}{0}+\vector{0}{3.8}}&=\magnitude{\vector{2.5}{3.8}}\\
    &=\sqrt{2.5^2+3.8^2}\approx\boxed{4.55\unit{km}}
\end{align*}

\subquestion{c}
Since this can be scenario can be set up as a right angle triangle, with the boat as the hypothesis' outer vertex, using trigenometry the angle can is determined as:
$$
    \tan\theta=\frac{o}{a}\leadsto\theta=\tan^{-1}\frac{2.5}{3.8}\approx33.3^\circ
$$

\question{A1}{11}{2}
\subquestion{a}
15 minutes corresponds to a 90 degree ($\frac{\pi}{2}\unit{rad}$) rotation on a clock, or one quarter of its perimeter.
Therefore the distance travelled by the tip of the pointer must be:
$$
    s=\frac{2\pi r}{4}=\frac{15\pi}{2}\approx23.6\unit{cm}
$$
The displacement however is the distance between the points $\point{0}{15}$ and $\point{15}{0}$, thus using Pythagoras' theorem\footnote{$a^2+b^2=c^2$}:
$$
    s=\sqrt{15^2+15^2}=\sqrt{450}=21.2\unit{cm}
$$

\subquestion{b}
Analagously to part \subquestion{a} but for the 180 degrees ($\pi\unit{rad}$) rotation resulting from 30 elapsed minutes:
\begin{align*}
    &s_{\text{distance}}=\frac{2\pi r}{2}=15\pi\approx47.1\unit{cm}\\
    &s_{\text{displacement}}=\sqrt{0^2+30^2}=30\unit{cm}
\end{align*}

\question{A1}{12}{3}
Since they are headed in completley oposite directions, $s_\text{Ada}+s_\text{Matt}=580\unit{m}$.
Ada's speed of $20\unit{km\,h^{-1}}$ is approximately $5.56\unit{m\,s^{-1}}$, as found by dividing through by 3.6.
Since $\displaystyle s=\int v\dd{t}$:
\begin{align*}
    v_\text{Ada}t+v_\text{Matt}t&=580\\
    5.56\times60+v_\text{Matt}\times60&=580\\
    \therefore v_\text{Matt}&\approx4.11\unit{m\,s^{-1}}\equiv\boxed{14.8\unit{km\,h^{-1}}}
\end{align*}

\question{A1}{12}{4}
First multiply the speed of light by the length of a light-year in seconds to obtain $1\unit{ly}$.
\begin{align*}
    1\unit{ly}&=3\times10^8\unit{m\,s^{-1}}\times(60\times60\times24\times365)\unit{s}\\
    &\approx9.46\times10^{15}
\end{align*}
Then, find the conversion between $1\unit{au}$ and one light-year.
$$
    1\unit{au}\approx\frac{1.5\times10^{11}}{9.46\times10^{15}}=1.59\times10^{-5}\unit{ly}
$$
Therefore,
$$
    5.5\times10^5\unit{au}\approx\boxed{8.72\unit{ly}}
$$

\question{A1}{14}{5}
\subquestion{a}
Calculating the speed between stations A and B:
$$
    v=\frac{1000}{80}=12.5\unit{m\,s^{-1}}
$$
Multiplying by $3.6$ to obtain the speed in $\unit{km\,h^{-1}}$:
$$
    v=12.5\times3.6=\boxed{45\unit{km\,h^{-1}}}
$$

\subquestion{b}
$\Delta y=1800-1000=800\unit{m}$

\subquestion{c}
The train travels $800\unit{m}$ in $60$ seconds, therefore, analagously to part \subquestion{a}:
$$
    v=\frac{800}{60}\approx13.3\unit{m\,s^{-1}}=\boxed{47.9\unit{km\,h^{-1}}}
$$

\question{A1}{14}{6}
\subquestion{a}
\subsubquestion{i}
$$
\frac{4\unit{m}}{10\unit{m\,s^{-1}}}=\boxed{0.4\unit{s}}
$$

\subsubquestion{ii}
The ball traveled for a total of $0.9\unit{s}$. If then, it took $0.4\unit{s}$ to the wall, then it took $0.9-0.4=0.5\unit{s}$.
The speed needed to travel $4\unit{m}$ in $0.5\unit{s}$ is:
$$
    \frac{4\unit{m}}{0.5\unit{s}}=\boxed{8\unit{m\,s^{-1}}}
$$

\subquestion{b}
\graphquestion

\question{A1}{16}{7}
\subquestion{a}
\subsubquestion{i}
Imagining a straight line tangent to the curve (since $\displaystyle v=\dv{s}{t}$), it goes roughly $6\unit{m}$ in $1\unit{s}$, therefore the velocity is $\boxed{6\unit{m\,s^{-1}}}$

\subsubquestion{ii}
The tangent line at $t=5\unit{s}$ goes approximately from $\point{2.5}{12.5}$ to $\point{10}{28}$, thus
\begin{align*}
    &\Delta s=28-12.5=15.5\unit{m}\\
    &\Delta t=10-2.5=7.5\unit{s}
\end{align*}
Therefore:
$$
    v=\frac{\Delta s}{\Delta t}=\frac{15.5}{7.5}\approx\boxed{2\unit{m\,s^{-1}}}
$$

\subquestion{b}
$$
    v=\frac{\Delta s}{\Delta t}=\frac{24}{12.5}=\boxed{1.92\unit{m\,s^{-1}}}
$$

\question{A1}{16}{8}
\subquestion{a}
The perimeter of the track is $s=\frac{2\pi r}{2}=25\pi\unit{m}$.
Therefore, the average speed is:
$$
    v=\frac{25\pi}{19}\approx\boxed{4.13\unit{m\,s^{-1}}}
$$

\subquestion{b}
For the velocity, we need the magnitude of the displacement which is $50\unit{m}$ since he went from one side of the circle to the other, one length of the diameeter ($2r$).
Therefore, the average velocity is:
$$
    v=\frac{50}{19}\approx\boxed{2.63\unit{m\,s^{-1}}}
$$

\question{A1}{21}{9}
Since $\displaystyle s=\int v\dd{t}$, the sum of the squares under the graph is the distance travelled.
Estimating using triangles ($A=\frac{bh}{2}$):
\begin{align*}
    s&=\frac{4\times1}{2}+\frac{2\times1}{2}+\frac{2\times3}{2}+11\\
    &=2+1+3+11=17\unit{m}
\end{align*}
The triangles in this case underestimate the area, so rounding up we get $20\unit{m}$, making the answer $\answer{B}$.

\question{A1}{21}{10}
The tangent line (since $\displaystyle a=\dv{v}{t}$) goes from roughly $\point{0}{4}$ to $\point{4}{8}$, thus:
\begin{align*}
    &\Delta v=8-4=4\\
    &\Delta t=4-0=4
\end{align*}
$$
    a=\frac{\Delta v}{\Delta t}=\frac{4}{4}=1\unit{m\,s^{-1}}
$$
Therefore the answer is \answer{A}.

\question{A1}{21}{11}
At $t=4\unit{s}$, the max speed has been reached.
Since the object was accelerated previously, the average speed is below the max line, thus the answer is either $\textrm{A}$ or $\textrm{C}$.
The tangent line (since $\displaystyle a=\dv{v}{t}$) at this point, however, is flat, meaning $a_{\text{instant}}=0$. 
Thus the answer is \answer{A}.

\question{A1}{21}{12}
If the velocity magnitude of the velocity is negative then it is moving in one direction, and if it is positive, then it moves in the oposite direction.
Therefore, each time the velocity changes sign, the object is changing direction.
The velocity changes sign twice, meaning the answer is \answer{B}.

\question{A1}{22}{13}
\subquestion{a}
Considering the absolute value of the graph, the velocity is first decreasing between $t=0\unit{s}$ and $1.5\unit{s}$ and then increasing between $t=1.5\unit{s}$ and $2.5\unit{s}$.
What happens however at $t=1.5\unit{s}$ is that, with the change of sign (vide \textbf{A1:21 Question~12}), is that the direction of travel changes.

\subquestion{b}
\subsubquestion{i}
The velocity decreases along a straight line, meaning acceleration is constant.
Calculating the slope of the line between 2 nice points:
$$
    a=\frac{\Delta v}{\Delta t}=\frac{-2}{1.5}\approx\boxed{-1.33\unit{m\,s^{-2}}}
$$

\subsubquestion{ii}
The positive and negatibe parts between $t=0.5\unit{s}$ and $2.5\unit{s}$ will cancel out, thus we only need to compute the area under the curve (since $\displaystyle s=\int v\dd{t}$) between $t=0\unit{s}$ and $0.5\unit{s}$.
To compute this we need to know the coordinates of the point $\point{0.5}{?}$.
Since we know $a\approx-1.33\unit{m\,s^{-2}}$, we can compute:
\begin{align*}
    v_{0.5}&\approx v_0-1.33\times0.5\\
    &=1.335\unit{m\,s^{-1}}
\end{align*}
Therefore summing the rectangle and triangle below the curve:
$$
    s=0.5\times1.335+\frac{(2-1.355)\times0.5}{2}\approx\boxed{0.834\unit{m}}
$$

\subquestion{c}
\graphquestion

\question{A1}{25}{14}
\subquestion{a}
Note that if the cart is returning to it's origin, then the distance travelled $s$ is 0.
Using $s=ut+\frac{1}{2}at^2$:
\begin{align*}
    0&=ut+\frac{1}{2}at^2\\
    &=3t-0.5\times1.8t^2\\
    \implies0.9t^2&=3t
\end{align*}
Dividing by $t$ discards the solution of $t=0$ (since division by 0 is not mathematically allowed), however this solution is trivvial (when the cart begins travelling it is at its origin):
\begin{align*}
    0.9t&=3\\
    \therefore t&=\frac{3}{0.9}\approx\boxed{3.33\unit{s}}
\end{align*}

\subquestion{b}
When the velocity $v$ is 0 the distance is maximum, since after the velocity changes sign (by passing zero), the object starts moving in the oposite direction.
Using $v^2=u^2+2as$:
\begin{align*}
    0&=3^2-2\times1.8s\\
    \implies s&=\frac{9}{3.6}=\boxed{2.5\unit{m}}
\end{align*}

\question{A1}{25}{15}
\subquestion{a}
Converting $100\unit{km\,h^{-1}}$ to $\unit{m\,s^{-1}}$ by dividing by 3.6 means:
$$
    v=\frac{100}{3.6}
$$
Then using $v=u+at$:
\begin{align*}
    \frac{100}{3.6}&=0+16a\\
    \implies a&=\frac{100}{3.6\times16}\approx\boxed{1.74\unit{m\,s^{-2}}}
\end{align*}

\subquestion{b}
Converting $250\unit{km\,h^{-1}}$ to $\unit{m\,s^{-1}}$ by dividing by 3.6 means:
$$
    v=\frac{250}{3.6}
$$
Using $v^2=u^2+2as$ and the acceleration from \subquestion{a} we derive:
\begin{align*}
    \frac{250^2}{3.6^2}&=0+2\times\frac{100}{57.6}s\\
    \implies s&=\frac{250^2\times57.6}{200\times3.6^2}\approx\boxed{1400\unit{m}}
\end{align*}

\question{A1}{25}{16}
Using $v^2=u^2+2as$:
\begin{align*}
    12^2&=u^2+2\times(-4.3)\times25\\
    \implies u&=\sqrt{12^2+8.6\times25}\approx\boxed{18.9\unit{m\,s^{-1}}}
\end{align*}

\question{A1}{25}{17}
\subquestion{a}
To reach maximum velocity and be able to stop in time, the train must accelerate the first half of the distance and then immediately start de-accelerating.
Therefore $s=360$.
Then using $v^2=u^2+2as$:
\begin{align*}
    v^2&=0+2\times1.3\times360\\
    \implies v&=\sqrt{720\times1.3}\approx\boxed{30.6\unit{m\,s^{-1}}}
\end{align*}

\subquestion{b}
The time is minimised if the velocity is maximised.
Since the travel is symmetrical, considering the first half ($s=360$) with maximum acceleration and using $s=ut+\frac{1}{2}at^2$:
\begin{align*}
    360&=0+\frac{1}{2}\times1.3\times t^2\\
    \implies t&=\sqrt{\frac{720}{1.3}}
\end{align*}
Since this was only half the distance, the total minimum travel time is $2t$:
$$
    2t=2\sqrt{\frac{720}{1.3}}\approx\boxed{47.1\unit{s}}
$$

\question{A1}{25}{18}
\subquestion{a}
Since we know the distance, initial velocity (0), and time, we can derive the acceleration through the SUVAT equaition $s=ut+\frac{1}{2}at^2$:
\begin{align*}
    12&=0+0.5a\times4^2\\
    \implies a&=\frac{12}{0.5\times4^2}=\frac{12}{8}=1.5\unit{m\,s^{-2}}
\end{align*}
Then using $v=u+at$ we find the velocity at $t=2$ is:
$$
    v=0+1.5\times2=3\unit{m\,s^{-1}}
$$
The answer matching this is \answer{D}.

\subquestion{b}
\graphquestion

\question{A1}{34}{19}
If both projectiles reach the same height, that means that the vertical component of the initial velocity was the same, since:
\begin{align*}
    {u_1}_yt-\frac{1}{2}gt^2&=\max y={u_2}_yt-\frac{1}{2}gt^2\\
    \implies{u_1}_y&={u_2}_y
\end{align*}
If the vertical component of the initial velocity is the same, then the time taken to reach the ground is the same.
This is because both are experiencing the same gravitational pull.
Thus the answer is \answer{B}.

\question{A1}{34}{20}
The horizontal distance travelled is proportional to the height $s\propto h$ since the time taken to hit the ground is proportional to the height $t_\text{final}\propto h$.
The time taken to hit the ground $s_y=-h$, with 0 initial vertical velocity, can be computed as:
\begin{align*}
    -h&=0-\frac{1}{2}gt_\text{final}^2\\
    \implies t_\text{final}&=\sqrt{\frac{2h}{g}}
\end{align*}
Note that the negative solution was discarded as it is physically impossible.
Thus we derive that:
$$
    t_\text{final}\propto\sqrt{h}
$$
Then, we compute that horizontal distance travelled (with initial horizontal velocity $u$, with no or negligible air resistance) as:
$$
    s=ut+0
$$
Therefore, the final distance travelled is $s$ evaluated at $t_\text{final}$, meaning:
$$
    s\propto\sqrt{h}
$$
Thus to reach a distance of $2s$, we need a height of $4h$, since $\sqrt{4h}=2\sqrt{h}$.
This means the answer is \answer{D}.

\question{A1}{34}{21}
Using $s=ut+\frac{1}{2}at^2$:
\begin{align*}
    s&=-4t-\frac{1}{2}gt^2\\
    &=-4\times1.9-\frac{g}{2}(1.9)^2\approx-25.3\unit{m}
\end{align*}
This is the displacement of the object over that time, so the tower's height is $h=-s=25.3\unit{m}$.
The closest answer is therefore \answer{D}.

\question{A1}{34}{22}
The time to hit the floor is given by, as derived in \textbf{A1:P34 Question 20}:
$$
    t_\text{final}=\sqrt{\frac{2h}{g}}
$$
The horizontal displacement is then given through $s=ut+\frac{1}{2}at^2$ (noting air resistance is negligible, providing 0 acceleration):
\begin{align*}
    s&=vt_\text{final}\\
    &=v\sqrt{\frac{2h}{g}}
\end{align*}
The answer is therefore \answer{C}.

\question{A1}{34}{23}
\subquestion{a}
The maximum height is reached when the vertical velocity is 0, since the changing of a sign changes the direction of motion, making the height of the ball strictly decrease post that point.
Therefore, using $v=u+at$:
\begin{align*}
    0&=u_y-gt_\text{max}\\
    u_y&=gt_\text{max}=0.9g\approx\boxed{8.84\unit{m\,s^{-1}}}
\end{align*}

\subquestion{b}
Noting there is no air resistance, the horizontal velocity can be computed using $s=ut+\frac{1}{2}at^2$ as:
\begin{align*}
    16&=0.9u_x+0\\
    \implies u_x&=\frac{16}{0.9}\approx17.8\unit{m\,s^{-1}}
\end{align*}
Using Pythagoras theorem, we get that the total initial velocity is:
$$
    u=\sqrt{u_x^2+u_y^2}=\sqrt{17.8^2+8.84^2}\approx\boxed{19.9\unit{m\,s^{-1}}}
$$

\subquestion{c}
Solving the equation $u\sin\theta=u_y$ (or $u\cos\theta=u_x$) gives:
\begin{align*}
    \sin\theta&\approx\frac{8.84}{19.9}\\
    \implies\theta&\approx\sin^{-1}\frac{8.84}{19.9}\approx0.4603\hdots\unit{rad}
\end{align*}
Converting to degrees (using $\theta_\text{deg}=\frac{180\theta_\text{rad}}{\pi}$):
$$
    \theta\approx\boxed{26.4^ \circ}
$$

\subquestion{d}
Using $s=ut+\frac{1}{2}at^2$:
\begin{align*}
    \max h&=u_yt_\text{max}-\frac{1}{2}gt_\text{max}^2\\
    &=8.84\times0.9-\frac{g}{2}0.9^2\approx\boxed{3.98\unit{m}}
\end{align*}

\question{A1}{34}{24}
Since there is 0 initial vertical velocity, the time to reach the net is (using $s=ut+\frac{1}{2}at^2$):
\begin{align*}
    0.9-2.7&=0-\frac{1}{2}gt_\text{final}^2\\
    \implies t_\text{final}&=\sqrt{\frac{2\times1.8}{g}}\qquad\text{(Negative solution not possible)}\\
    &\approx0.605\unit{s}
\end{align*}
Then to derive the initial horizontal velocity (with assumed 0 acceleration), we use $s=ut+\frac{1}{2}at^2$ again:
\begin{align*}
    12&=u_xt_\text{final}+0\\
    \implies u_x&\approx\frac{12}{0.605}\approx\boxed{19.8\unit{m\,s^{-1}}}
\end{align*}

\question{A1}{34}{25}
\subquestion{a}
The initial vertical velocity $u\sin\theta$ is approximately $0.628\unit{m\,s^{-1}}$.
Then using $s=ut+\frac{1}{2}at^2$ and checking for approximate equality at $t=0.3\unit{s}$:
\begin{align*}
    -0.25&\approx0.628\times0.3-\frac{1}{2}g(0.3)^2\\
    -0.25&\approx-0.2535\quad\color{Green}\checkmark
\end{align*}

\subquestion{b}
\subsubquestion{i}
The initial horizontal velocity $u\cos\theta$ is approximately $8.98\unit{m\,s^{-1}}$.
Using $s=ut+\frac{1}{2}at^2$ with assumed 0 acceleration:
$$
    s\approx8.98\times0.3+0=\boxed{2.69\unit{m}}
$$

\subsubquestion{ii}
Only the vertical velocity is changing (due to the assumed 0 horizontal acceleration).
Therefore, using $v=u+at$:
$$
    v_\text{final}\approx0.628-0.3g=-2.32\unit{m\,s^{-1}}
$$
Then using Pythagoras theorem to calculate the total final velocity:
$$
    v=\sqrt{v_x^2+v_y^2}=\sqrt{8.98^2+(-2.32)^2}\approx\boxed{9.27\unit{m\,s^{-1}}}
$$

\subquestion{c}
\subsubquestion{i}
\graphquestion

\subsubquestion{ii}
\graphquestion

\subsection{Subtopic A.2 – Forces and momentum}
\question{A2}{46}{1}
Disregarding the initial moment, in which force is applied to propell the object, the only force acting on the object is gravity pull.
Gravitational pull is always directed downwards, meaning the answer is \answer{D}.
Note that what is 0 isn't the force, but the velocity (since the sign is changing).

\question{A2}{46}{2}
Ignore all vertical movement.
Using $v^2=u^2+2as$ (with $v=0$ since the pellet comes to rest):
\begin{align*}
    0&=200^2+2a\times0.1\\
    \implies a&=-\frac{40000}{0.2}=-200000\unit{m\,s^{-2}}
\end{align*}
Then using Newton's second law ($\force\newton{2}ma$), and only considering the magnitude (ignoring the $-$ sign):
$$
    \force=0.002\times200000=400\unit{N}
$$
Thus, the answer is \answer{C}.

\question{A2}{47}{3}
The average acceleration is computed using $v=u+at$ as:
\begin{align*}
    15&=0+0.01a\\
    \implies a&=\frac{15}{0.01}=1500\unit{m\,s^{-2}}
\end{align*}
Then using Newton's second law ($\force\newton{2}ma$):
$$
    \force=0.058\times1500=\boxed{87\unit{N}}
$$

\question{A2}{47}{4}
\subquestion{a}
First convert the speed units to $\unit{m\,s^{-1}}$ by dividing through by 3.6:
$$
    \left\{\begin{matrix}
        v=12.5\unit{m\,s^{-1}}\\
        u=22.\bar{2}\unit{m\,s^{-1}}
    \end{matrix}\right.
$$
Then using $v^2=u^2+2as$:
\begin{align*}
    12.5^2&=22.\bar{2}^2+36a\\
    \implies a&=\frac{12.5^2-22.\bar{2}^2}{36}\approx-9.35\unit{m\,s^{-2}}
\end{align*}
Therefore, the average force (only considering magnitude) must be:
$$
    \force\newton{2}ma\approx1200\times9.35\approx\boxed{11220\unit{N}}
$$

\subquestion{b}
Using $v=u+at$:
\begin{align*}
    12.5&\approx22.\bar{2}-9.35t\\
    \implies t&\approx\frac{12.5-22.\bar{2}}{-9.35}\approx\boxed{1.04\unit{s}}
\end{align*}

\question{A2}{47}{5}
\subquestion{a}
Noting that Newton's second law states $\force\newton{2}ma$, we see that:
\begin{align*}
    a=\frac{\force}{m}=\frac{400}{11000}=\frac{2}{55}\unit{m\,s^{-2}}
\end{align*}
Then using $v=u+at$ (and noting that since it was stationary at $t=0$, $u=0$):
$$
    v=0+\frac{2}{55}\times8\approx\boxed{0.291\unit{m\,s^{-1}}}
$$

\subquestion{b}
Using $s=ut+\frac{1}{2}at^2$:
\begin{align*}
    s&=0+\frac{1}{2}\frac{2}{55}\times8^2\\
    &=\frac{64}{55}\approx\boxed{1.16\unit{m}}
\end{align*}

\question{A2}{47}{6}
\subquestion{a}
The force acting on the electron is only acting along the vertical.
Thus, if there is no force acting horizontally, then there is no acceleration horizontally.
Without acceleration, the velocity remains constant (Newton's first law).

\subquestion{b}
\subsubquestion{i}
The time taken for the electron to travel the 25$\unit{cm}$ is given by:
$$
    t=\frac{0.25\unit{m}}{8\times10^6\unit{m\,s^{-1}}}=3.125\times10^{-8}
$$
We then calculate the acceleration acting on the electron using Newton's second law:
$$
    a\newton{2}\frac{\force}{m}=\frac{6.4\times10^{-17}}{9.11\times10^{-31}}=\frac{6.4}{9.11}\times10^{14}\unit{m\,s^{-2}}
$$
Finally using $s=ut+\frac{1}{2}at^2$:
\begin{align*}
    s&=0+\frac{1}{2}\frac{6.4}{9.11}\times10^{14}\times\qty(3.125\times10^{-8})^2\\
    &\approx\boxed{3.43\unit{cm}}
\end{align*}

\subsubquestion{ii}
Using $v=u+at$ we get a vertical velocity of:
\begin{align*}
    v_y&=0+\frac{6.4}{9.11}\times10^{14}\times3.125\times10^{-8}\\
    &\approx2.2\times10^6\unit{m\,s^{-1}}
\end{align*}
Then using $\displaystyle\tan\theta=\frac{v_y}{v_x}$ ($v_x$ is the velocity stated in the question):
\begin{align*}
    \theta=\tan^{-1}\frac{2.2\times10^6}{8\times10^6}\approx\boxed{15.4^\circ}
\end{align*}

\question{A2}{47}{7}
\subquestion{a}
\graphquestion
If there is no acceleration, then no force other than gravity is acting on the person.
Thus the scale, measuring force, will show just the weight force:
$$
    \force_g\newton{2}mg=75\times9.82\approx\boxed{737\unit{N}}
$$

\subquestion{b}
\graphquestion
The force driving the elevator upwards will push against the person in the elevator.
By Newton's third law, the person will then experience an equal force, but in the oposite direction.
Since the force on the elevator acts upwards, the force on the person acts downwards with an acceleration of $2\unit{m\,s^{-2}}$.

The magnitude of the force acting downwards on the person is given as:
$$
    \force_\downarrow\newton{2}ma=75\times2=150\unit{N}
$$

Therefore the magnitude of the resultant force $\force_R$ is given as ($\force_g$ was computed in \subquestion{a}):
$$
    \force_R=\force_g+\force_\downarrow\approx737+150=\boxed{887\unit{N}}
$$

\question{A2}{47}{8}
\subquestion{a}
Let the force of tension in each string be denoted $\force_t$.
Combined, the vertical component of both string's tension ($\force_t\sin\theta$) force, $2\force_t\sin\theta$, must balance the weight force of the object, $\force_g$.
The weight force of the object is given as:
$$
    \force_g\newton{2}mg=2g\approx19.6\unit{N}
$$
Then to derive the angle $\theta$, splitting the 150$^\circ$ in to two gives 75$^\circ$.
Then, in consideration of the right angle triangle on the side towards the weight, we get the angle $\theta=90-75=15^\circ$.
Finally, we solve the equation:
\begin{align*}
    2\force_t\sin15^\circ&=\force_g\\
    \force_t&=\frac{19.6}{2\sin15^\circ}\approx\boxed{37.9\unit{N}}
\end{align*}

\subquestion{b}
The threads can only support a certain amount of tension.
As the angle (let this be called $\phi$) between the strings increases, the angle $\theta$ decreases.
This is because (as demonstrated in \subquestion{a}):
$$
    \theta=90-\frac{\phi}{2}
$$
The equation for the force of tension is proportional to (as demonstrated in \subquestion{a}):
\begin{align*}
    \force_t&\propto\frac{1}{\sin\theta}=\frac{1}{\sin90-\frac{\phi}{2}}\\
    &=\frac{1}{\cos\frac{\phi}{2}}\\
    \therefore\force_t&\propto\frac{1}{\cos\phi}
\end{align*}
As $\phi$ increases towards 180$^\circ$, $\cos\phi$ increases as well, meaning the force of tension, too, increases.
Thus be increasing the angle, the threads are more likely to break.

\end{A}