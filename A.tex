\backgroundsetup
{
    scale=1,
    angle=0,
    opacity=1,
    contents={\begin{tikzpicture}[remember picture,overlay]
        \path [fill=Space] (-\paperwidth, 0.375\paperheight)rectangle (\paperwidth, \paperheight); 
    \end{tikzpicture}}
}
\BgThispage

\begin{A}
\section{Topic A – Space, time and motion}
\subsection{Subtopic A.1 – Kinematics}
\question{A1}{11}{1}
\subquestion{a}
The total distance is given as the sum of distance traveled during each translation, $2.5+3.8=6.3\unit{km}$

\subquestion{b}
Displacement is a vector quantity, thus taking the magnitude of the sum of the two vector displacements caused by the movement:
\begin{align*}
    \magnitude{\vector{2.5}{0}+\vector{0}{3.8}}&=\magnitude{\vector{2.5}{3.8}}\\
    &=\sqrt{2.5^2+3.8^2}\approx\boxed{4.55\unit{km}}
\end{align*}

\subquestion{c}
Since this can be scenario can be set up as a right angle triangle, with the boat as the hypothesis' outer vertex, using trigenometry the angle can is determined as:
$$
    \tan\theta=\frac{o}{a}\leadsto\theta=\tan^{-1}\frac{2.5}{3.8}\approx33.3^\circ
$$

\question{A1}{11}{2}
\subquestion{a}
15 minutes corresponds to a 90 degree ($\frac{\pi}{2}\unit{rad}$) rotation on a clock, or one quarter of its perimeter.
Therefore the distance travelled by the tip of the pointer must be:
$$
    s=\frac{2\pi r}{4}=\frac{15\pi}{2}\approx23.6\unit{cm}
$$
The displacement however is the distance between the points $\point{0}{15}$ and $\point{15}{0}$, thus using Pythagoras' theorem\footnote{$a^2+b^2=c^2$}:
$$
    s=\sqrt{15^2+15^2}=\sqrt{450}=21.2\unit{cm}
$$

\subquestion{b}
Analagously to part \subquestion{a} but for the 180 degrees ($\pi\unit{rad}$) rotation resulting from 30 elapsed minutes:
\begin{align*}
    &s_{\text{distance}}=\frac{2\pi r}{2}=15\pi\approx47.1\unit{cm}\\
    &s_{\text{displacement}}=\sqrt{0^2+30^2}=30\unit{cm}
\end{align*}

\question{A1}{12}{3}
Since they are headed in completley oposite directions, $s_\text{Ada}+s_\text{Matt}=580\unit{m}$.
Ada's speed of $20\unit{km\,h^{-1}}$ is approximately $5.56\unit{m\,s^{-1}}$, as found by dividing through by 3.6.
Since $\displaystyle s=\int v\dd{t}$:
\begin{align*}
    v_\text{Ada}t+v_\text{Matt}t&=580\\
    5.56\times60+v_\text{Matt}\times60&=580\\
    \therefore v_\text{Matt}&\approx4.11\unit{m\,s^{-1}}\equiv\boxed{14.8\unit{km\,h^{-1}}}
\end{align*}

\question{A1}{12}{4}
First multiply the speed of light by the length of a light-year in seconds to obtain $1\unit{ly}$.
\begin{align*}
    1\unit{ly}&=3\times10^8\unit{m\,s^{-1}}\times(60\times60\times24\times365)\unit{s}\\
    &\approx9.46\times10^{15}
\end{align*}
Then, find the conversion between $1\unit{au}$ and one light-year.
$$
    1\unit{au}\approx\frac{1.5\times10^{11}}{9.46\times10^{15}}=1.59\times10^{-5}\unit{ly}
$$
Therefore,
$$
    5.5\times10^5\unit{au}\approx\boxed{8.72\unit{ly}}
$$

\question{A1}{14}{5}
\subquestion{a}
Calculating the speed between stations A and B:
$$
    v=\frac{1000}{80}=12.5\unit{m\,s^{-1}}
$$
Multiplying by $3.6$ to obtain the speed in $\unit{km\,h^{-1}}$:
$$
    v=12.5\times3.6=\boxed{45\unit{km\,h^{-1}}}
$$

\subquestion{b}
$\Delta y=1800-1000=800\unit{m}$

\subquestion{c}
The train travels $800\unit{m}$ in $60$ seconds, therefore, analagously to part \subquestion{a}:
$$
    v=\frac{800}{60}\approx13.3\unit{m\,s^{-1}}=\boxed{47.9\unit{km\,h^{-1}}}
$$

\question{A1}{14}{6}
\subquestion{a}
\subsubquestion{i}
$$
\frac{4\unit{m}}{10\unit{m\,s^{-1}}}=\boxed{0.4\unit{s}}
$$

\subsubquestion{ii}
The ball traveled for a total of $0.9\unit{s}$. If then, it took $0.4\unit{s}$ to the wall, then it took $0.9-0.4=0.5\unit{s}$.
The speed needed to travel $4\unit{m}$ in $0.5\unit{s}$ is:
$$
    \frac{4\unit{m}}{0.5\unit{s}}=\boxed{8\unit{m\,s^{-1}}}
$$

\subquestion{b}
\graphquestion

\question{A1}{16}{7}
\subquestion{a}
\subsubquestion{i}
Imagining a straight line tangent to the curve (since $\displaystyle v=\dv{s}{t}$), it goes roughly $6\unit{m}$ in $1\unit{s}$, therefore the velocity is $\boxed{6\unit{m\,s^{-1}}}$

\subsubquestion{ii}
The tangent line at $t=5\unit{s}$ goes approximately from $\point{2.5}{12.5}$ to $\point{10}{28}$, thus
\begin{align*}
    &\Delta s=28-12.5=15.5\unit{m}\\
    &\Delta t=10-2.5=7.5\unit{s}
\end{align*}
Therefore:
$$
    v=\frac{\Delta s}{\Delta t}=\frac{15.5}{7.5}\approx\boxed{2\unit{m\,s^{-1}}}
$$

\subquestion{b}
$$
    v=\frac{\Delta s}{\Delta t}=\frac{24}{12.5}=\boxed{1.92\unit{m\,s^{-1}}}
$$

\question{A1}{16}{8}
\subquestion{a}
The perimeter of the track is $s=\frac{2\pi r}{2}=25\pi\unit{m}$.
Therefore, the average speed is:
$$
    v=\frac{25\pi}{19}\approx\boxed{4.13\unit{m\,s^{-1}}}
$$

\subquestion{b}
For the velocity, we need the magnitude of the displacement which is $50\unit{m}$ since he went from one side of the circle to the other, one length of the diameeter ($2r$).
Therefore, the average velocity is:
$$
    v=\frac{50}{19}\approx\boxed{2.63\unit{m\,s^{-1}}}
$$

\question{A1}{21}{9}
Since $\displaystyle s=\int v\dd{t}$, the sum of the squares under the graph is the distance travelled.
Estimating using triangles ($A=\frac{bh}{2}$):
\begin{align*}
    s&=\frac{4\times1}{2}+\frac{2\times1}{2}+\frac{2\times3}{2}+11\\
    &=2+1+3+11=17\unit{m}
\end{align*}
The triangles in this case underestimate the area, so rounding up we get $20\unit{m}$, making the answer $\answer{B}$.

\question{A1}{21}{10}
The tangent line (since $\displaystyle a=\dv{v}{t}$) goes from roughly $\point{0}{4}$ to $\point{4}{8}$, thus:
\begin{align*}
    &\Delta v=8-4=4\\
    &\Delta t=4-0=4
\end{align*}
$$
    a=\frac{\Delta v}{\Delta t}=\frac{4}{4}=1\unit{m\,s^{-1}}
$$
Therefore the answer is \answer{A}.

\question{A1}{21}{11}
At $t=4\unit{s}$, the max speed has been reached.
Since the object was accelerated previously, the average speed is below the max line, thus the answer is either $\textrm{A}$ or $\textrm{C}$.
The tangent line (since $\displaystyle a=\dv{v}{t}$) at this point, however, is flat, meaning $a_{\text{instant}}=0$. 
Thus the answer is \answer{A}.

\question{A1}{21}{12}
If the velocity magnitude of the velocity is negative then it is moving in one direction, and if it is positive, then it moves in the oposite direction.
Therefore, each time the velocity changes sign, the object is changing direction.
The velocity changes sign twice, meaning the answer is \answer{B}.

\question{A1}{22}{13}
\subquestion{a}
Considering the absolute value of the graph, the velocity is first decreasing between $t=0\unit{s}$ and $1.5\unit{s}$ and then increasing between $t=1.5\unit{s}$ and $2.5\unit{s}$.
What happens however at $t=1.5\unit{s}$ is that, with the change of sign (vide \textbf{A1:21 Question~12}), is that the direction of travel changes.

\subquestion{b}
\subsubquestion{i}
The velocity decreases along a straight line, meaning acceleration is constant.
Calculating the slope of the line between 2 nice points:
$$
    a=\frac{\Delta v}{\Delta t}=\frac{-2}{1.5}\approx\boxed{-1.33\unit{m\,s^{-2}}}
$$

\subsubquestion{ii}
The positive and negatibe parts between $t=0.5\unit{s}$ and $2.5\unit{s}$ will cancel out, thus we only need to compute the area under the curve (since $\displaystyle s=\int v\dd{t}$) between $t=0\unit{s}$ and $0.5\unit{s}$.
To compute this we need to know the coordinates of the point $\point{0.5}{?}$.
Since we know $a\approx-1.33\unit{m\,s^{-2}}$, we can compute:
\begin{align*}
    v_{0.5}&\approx v_0-1.33\times0.5\\
    &=1.335\unit{m\,s^{-1}}
\end{align*}
Therefore summing the rectangle and triangle below the curve:
$$
    s=0.5\times1.335+\frac{(2-1.355)\times0.5}{2}\approx\boxed{0.834\unit{m}}
$$

\subquestion{c}
\graphquestion

\question{A1}{25}{14}
\subquestion{a}
Note that if the cart is returning to it's origin, then the distance travelled $s$ is 0.
Using $s=ut+\frac{1}{2}at^2$:
\begin{align*}
    0&=ut+\frac{1}{2}at^2\\
    &=3t-0.5\times1.8t^2\\
    \implies0.9t^2&=3t
\end{align*}
Dividing by $t$ discards the solution of $t=0$ (since division by 0 is not mathematically allowed), however this solution is trivvial (when the cart begins travelling it is at its origin):
\begin{align*}
    0.9t&=3\\
    \therefore t=\frac{3}{0.9}\approx\boxed{3.33\unit{s}}
\end{align*}

\subquestion{b}
When the velocity $v$ is 0 the distance is maximum, since after the velocity changes sign (by passing zero), the object starts moving in the oposite direction.
Using $v^2=u^2+2as$:
\begin{align*}
    0&=3^2-2\times1.8s\\
    \implies s&=\frac{9}{3.6}=\boxed{2.5\unit{m}}
\end{align*}

\question{A1}{25}{15}
\subquestion{a}
Converting $100\unit{km\,h^{-1}}$ to $\unit{m\,s^{-1}}$ by dividing by 3.6 means:
$$
    v=\frac{100}{3.6}
$$
Then using $v=u+at$:
\begin{align*}
    \frac{100}{3.6}&=0+16a\\
    \implies a&=\frac{100}{3.6\times16}\approx\boxed{1.74\unit{m\,s^{-2}}}
\end{align*}

\subquestion{b}
Converting $250\unit{km\,h^{-1}}$ to $\unit{m\,s^{-1}}$ by dividing by 3.6 means:
$$
    v=\frac{250}{3.6}
$$
Using $v^2=u^2+2as$ and the acceleration from \subquestion{a} we derive:
\begin{align*}
    \frac{250^2}{3.6^2}&=0+2\times\frac{100}{57.6}s\\
    \implies s&=\frac{250^2\times57.6}{200\times3.6^2}\approx\boxed{1400\unit{m}}
\end{align*}

\question{A1}{25}{16}
Using $v^2=u^2+2as$:
\begin{align*}
    12^2&=u^2+2\times(-4.3)\times25\\
    \implies u&=\sqrt{12^2+8.6\times25}\approx\boxed{18.9\unit{m\,s^{-1}}}
\end{align*}

\question{A1}{25}{17}
\subquestion{a}
To reach maximum velocity and be able to stop in time, the train must accelerate the first half of the distance and then immediately start de-accelerating.
Therefore $s=360$.
Then using $v^2=u^2+2as$:
\begin{align*}
    v^2&=0+2\times1.3\times360\\
    \implies v&=\sqrt{720\times1.3}\approx\boxed{30.6\unit{m\,s^{-1}}}
\end{align*}

\subquestion{b}
The time is minimised if the velocity is maximised.
Since the travel is symmetrical, considering the first half ($s=360$) with maximum acceleration and using $s=ut+\frac{1}{2}at^2$:
\begin{align*}
    360&=0+\frac{1}{2}\times1.3\times t^2\\
    \implies t&=\sqrt{\frac{720}{1.3}}
\end{align*}
Since this was only half the distance, the total minimum travel time is $2t$:
$$
    2t=2\sqrt{\frac{720}{1.3}}\approx\boxed{47.1\unit{s}}
$$

\question{A1}{25}{18}
\subquestion{a}
Since we know the distance, initial velocity (0), and time, we can derive the acceleration through the SUVAT equaition $s=ut+\frac{1}{2}at^2$:
\begin{align*}
    12&=0+0.5a\times4^2\\
    \implies a&=\frac{12}{0.5\times4^2}=\frac{12}{8}=1.5\unit{m\,s^{-2}}
\end{align*}
Then using $v=u+at$ we find the velocity at $t=2$ is:
$$
    v=0+1.5\times2=3\unit{m\,s^{-1}}
$$
The answer matching this is \answer{D}.

\subquestion{b}
\graphquestion

\end{A}