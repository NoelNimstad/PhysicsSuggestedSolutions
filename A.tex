\backgroundsetup
{
    scale=1,
    angle=0,
    opacity=1,
    contents={\begin{tikzpicture}[remember picture,overlay]
        \path [fill=Space] (-\paperwidth, 0.375\paperheight)rectangle (\paperwidth, \paperheight); 
    \end{tikzpicture}}
}
\BgThispage

\begin{A}
\section{Topic A – Space, time and motion}
\subsection{Subtopic A.1 – Kinematics}
\question{A1}{11}{1}
\subquestion{a}
The total distance is given as the sum of distance traveled during each translation, $2.5+3.8=6.3\unit{km}$

\subquestion{b}
Displacement is a vector quantity, thus taking the magnitude of the sum of the two vector displacements caused by the movement:
\begin{align*}
    \magnitude{\vector{2.5}{0}+\vector{0}{3.8}}&=\magnitude{\vector{2.5}{3.8}}\\
    &=\sqrt{2.5^2+3.8^2}\approx\boxed{4.55\unit{km}}
\end{align*}

\subquestion{c}
Since this can be scenario can be set up as a right angle triangle, with the boat as the hypothesis' outer vertex, using trigenometry the angle can is determined as:
$$
    \tan\theta=\frac{o}{a}\leadsto\theta=\tan^{-1}\frac{2.5}{3.8}\approx33.3^\circ
$$

\question{A1}{11}{2}
\subquestion{a}
15 minutes corresponds to a 90 degree ($\frac{\pi}{2}\unit{rad}$) rotation on a clock, or one quarter of its perimeter.
Therefore the distance travelled by the tip of the pointer must be:
$$
    s=\frac{2\pi r}{4}=\frac{15\pi}{2}\approx23.6\unit{cm}
$$
The displacement however is the distance between the points $\point{0}{15}$ and $\point{15}{0}$, thus using Pythagoras' theorem\footnote{$a^2+b^2=c^2$}:
$$
    s=\sqrt{15^2+15^2}=\sqrt{450}=21.2\unit{cm}
$$

\subquestion{b}
Analagously to part \subquestion{a} but for the 180 degrees ($\pi\unit{rad}$) rotation resulting from 30 elapsed minutes:
\begin{align*}
    &s_{\text{distance}}=\frac{2\pi r}{2}=15\pi\approx47.1\unit{cm}\\
    &s_{\text{displacement}}=\sqrt{0^2+30^2}=30\unit{cm}
\end{align*}

\question{A1}{12}{3}
Since they are headed in completley oposite directions, $s_\text{Ada}+s_\text{Matt}=580\unit{m}$.
Ada's speed of $20\unit{km\,h^{-1}}$ is approximately $5.56\unit{m\,s^{-1}}$, as found by dividing through by 3.6.
Since $\displaystyle s=\int v\dd{t}$:
\begin{align*}
    v_\text{Ada}t+v_\text{Matt}t&=580\\
    5.56\times60+v_\text{Matt}\times60&=580\\
    \therefore v_\text{Matt}&\approx4.11\unit{m\,s^{-1}}\equiv\boxed{14.8\unit{km\,h^{-1}}}
\end{align*}

\question{A1}{12}{4}
First multiply the speed of light by the length of a light-year in seconds to obtain $1\unit{ly}$.
\begin{align*}
    1\unit{ly}&=3\times10^8\unit{m\,s^{-1}}\times(60\times60\times24\times365)\unit{s}\\
    &\approx9.46\times10^{15}
\end{align*}
Then, find the conversion between $1\unit{au}$ and one light-year.
$$
    1\unit{au}\approx\frac{1.5\times10^{11}}{9.46\times10^{15}}=1.59\times10^{-5}\unit{ly}
$$
Therefore,
$$
    5.5\times10^5\unit{au}\approx\boxed{8.72\unit{ly}}
$$

\question{A1}{14}{5}
\subquestion{a}
Calculating the speed between stations A and B:
$$
    v=\frac{1000}{80}=12.5\unit{m\,s^{-1}}
$$
Multiplying by $3.6$ to obtain the speed in $\unit{km\,h^{-1}}$:
$$
    v=12.5\times3.6=\boxed{45\unit{km\,h^{-1}}}
$$

\subquestion{b}
$\Delta y=1800-1000=800\unit{m}$

\subquestion{c}
The train travels $800\unit{m}$ in $60$ seconds, therefore, analagously to part \subquestion{a}:
$$
    v=\frac{800}{60}\approx13.3\unit{m\,s^{-1}}=\boxed{47.9\unit{km\,h^{-1}}}
$$

\question{A1}{14}{6}
\subquestion{a}
\subsubquestion{i}
$$
\frac{4\unit{m}}{10\unit{m\,s^{-1}}}=\boxed{0.4\unit{s}}
$$

\subsubquestion{ii}
The ball traveled for a total of $0.9\unit{s}$. If then, it took $0.4\unit{s}$ to the wall, then it took $0.9-0.4=0.5\unit{s}$.
The speed needed to travel $4\unit{m}$ in $0.5\unit{s}$ is:
$$
    \frac{4\unit{m}}{0.5\unit{s}}=\boxed{8\unit{m\,s^{-1}}}
$$

\subquestion{b}
\graphquestion

\question{A1}{16}{7}
\subquestion{a}
\subsubquestion{i}
Imagining a straight line tangent to the curve (since $\displaystyle v=\dv{s}{t}$), it goes roughly $6\unit{m}$ in $1\unit{s}$, therefore the velocity is $\boxed{6\unit{m\,s^{-1}}}$

\subsubquestion{ii}
The tangent line at $t=5\unit{s}$ goes approximately from $\point{2.5}{12.5}$ to $\point{10}{28}$, thus
\begin{align*}
    &\Delta s=28-12.5=15.5\unit{m}\\
    &\Delta t=10-2.5=7.5\unit{s}
\end{align*}
Therefore:
$$
    v=\frac{\Delta s}{\Delta t}=\frac{15.5}{7.5}\approx\boxed{2\unit{m\,s^{-1}}}
$$

\subquestion{b}
$$
    v=\frac{\Delta s}{\Delta t}=\frac{24}{12.5}=\boxed{1.92\unit{m\,s^{-1}}}
$$

\question{A1}{16}{8}
\subquestion{a}
The perimeter of the track is $s=\frac{2\pi r}{2}=25\pi\unit{m}$.
Therefore, the average speed is:
$$
    v=\frac{25\pi}{19}\approx\boxed{4.13\unit{m\,s^{-1}}}
$$

\subquestion{b}
For the velocity, we need the magnitude of the displacement which is $50\unit{m}$ since he went from one side of the circle to the other, one length of the diameeter ($2r$).
Therefore, the average velocity is:
$$
    v=\frac{50}{19}\approx\boxed{2.63\unit{m\,s^{-1}}}
$$

\question{A1}{21}{9}
Since $\displaystyle s=\int v\dd{t}$, the sum of the squares under the graph is the distance travelled.
Estimating using triangles ($A=\frac{bh}{2}$):
\begin{align*}
    s&=\frac{4\times1}{2}+\frac{2\times1}{2}+\frac{2\times3}{2}+11\\
    &=2+1+3+11=17\unit{m}
\end{align*}
The triangles in this case underestimate the area, so rounding up we get $20\unit{m}$, making the answer $\answer{B}$.

\question{A1}{21}{10}
The tangent line (since $\displaystyle a=\dv{v}{t}$) goes from roughly $\point{0}{4}$ to $\point{4}{8}$, thus:
\begin{align*}
    &\Delta v=8-4=4\\
    &\Delta t=4-0=4
\end{align*}
$$
    a=\frac{\Delta v}{\Delta t}=\frac{4}{4}=1\unit{m\,s^{-1}}
$$
Therefore the answer is \answer{A}.

\question{A1}{21}{11}
At $t=4\unit{s}$, the max speed has been reached.
Since the object was accelerated previously, the average speed is below the max line, thus the answer is either $\textrm{A}$ or $\textrm{C}$.
The tangent line (since $\displaystyle a=\dv{v}{t}$) at this point, however, is flat, meaning $a_{\text{instant}}=0$. 
Thus the answer is \answer{A}.

\question{A1}{21}{12}
If the velocity magnitude of the velocity is negative then it is moving in one direction, and if it is positive, then it moves in the oposite direction.
Therefore, each time the velocity changes sign, the object is changing direction.
The velocity changes sign twice, meaning the answer is \answer{B}.

\question{A1}{22}{13}
\subquestion{a}
Considering the absolute value of the graph, the velocity is first decreasing between $t=0\unit{s}$ and $1.5\unit{s}$ and then increasing between $t=1.5\unit{s}$ and $2.5\unit{s}$.
What happens however at $t=1.5\unit{s}$ is that, with the change of sign (vide \textbf{A1:21 Question~12}), is that the direction of travel changes.

\subquestion{b}
\subsubquestion{i}
The velocity decreases along a straight line, meaning acceleration is constant.
Calculating the slope of the line between 2 nice points:
$$
    a=\frac{\Delta v}{\Delta t}=\frac{-2}{1.5}\approx\boxed{-1.33\unit{m\,s^{-2}}}
$$

\subsubquestion{ii}
The positive and negatibe parts between $t=0.5\unit{s}$ and $2.5\unit{s}$ will cancel out, thus we only need to compute the area under the curve (since $\displaystyle s=\int v\dd{t}$) between $t=0\unit{s}$ and $0.5\unit{s}$.
To compute this we need to know the coordinates of the point $\point{0.5}{?}$.
Since we know $a\approx-1.33\unit{m\,s^{-2}}$, we can compute:
\begin{align*}
    v_{0.5}&\approx v_0-1.33\times0.5\\
    &=1.335\unit{m\,s^{-1}}
\end{align*}
Therefore summing the rectangle and triangle below the curve:
$$
    s=0.5\times1.335+\frac{(2-1.355)\times0.5}{2}\approx\boxed{0.834\unit{m}}
$$

\subquestion{c}
\graphquestion

\question{A1}{25}{14}
\subquestion{a}
Note that if the cart is returning to it's origin, then the distance travelled $s$ is 0.
Using $s=ut+\frac{1}{2}at^2$:
\begin{align*}
    0&=ut+\frac{1}{2}at^2\\
    &=3t-0.5\times1.8t^2\\
    \implies0.9t^2&=3t
\end{align*}
Dividing by $t$ discards the solution of $t=0$ (since division by 0 is not mathematically allowed), however this solution is trivvial (when the cart begins travelling it is at its origin):
\begin{align*}
    0.9t&=3\\
    \therefore t&=\frac{3}{0.9}\approx\boxed{3.33\unit{s}}
\end{align*}

\subquestion{b}
When the velocity $v$ is 0 the distance is maximum, since after the velocity changes sign (by passing zero), the object starts moving in the oposite direction.
Using $v^2=u^2+2as$:
\begin{align*}
    0&=3^2-2\times1.8s\\
    \implies s&=\frac{9}{3.6}=\boxed{2.5\unit{m}}
\end{align*}

\question{A1}{25}{15}
\subquestion{a}
Converting $100\unit{km\,h^{-1}}$ to $\unit{m\,s^{-1}}$ by dividing by 3.6 means:
$$
    v=\frac{100}{3.6}
$$
Then using $v=u+at$:
\begin{align*}
    \frac{100}{3.6}&=0+16a\\
    \implies a&=\frac{100}{3.6\times16}\approx\boxed{1.74\unit{m\,s^{-2}}}
\end{align*}

\subquestion{b}
Converting $250\unit{km\,h^{-1}}$ to $\unit{m\,s^{-1}}$ by dividing by 3.6 means:
$$
    v=\frac{250}{3.6}
$$
Using $v^2=u^2+2as$ and the acceleration from \subquestion{a} we derive:
\begin{align*}
    \frac{250^2}{3.6^2}&=0+2\times\frac{100}{57.6}s\\
    \implies s&=\frac{250^2\times57.6}{200\times3.6^2}\approx\boxed{1400\unit{m}}
\end{align*}

\question{A1}{25}{16}
Using $v^2=u^2+2as$:
\begin{align*}
    12^2&=u^2+2\times(-4.3)\times25\\
    \implies u&=\sqrt{12^2+8.6\times25}\approx\boxed{18.9\unit{m\,s^{-1}}}
\end{align*}

\question{A1}{25}{17}
\subquestion{a}
To reach maximum velocity and be able to stop in time, the train must accelerate the first half of the distance and then immediately start de-accelerating.
Therefore $s=360$.
Then using $v^2=u^2+2as$:
\begin{align*}
    v^2&=0+2\times1.3\times360\\
    \implies v&=\sqrt{720\times1.3}\approx\boxed{30.6\unit{m\,s^{-1}}}
\end{align*}

\subquestion{b}
The time is minimised if the velocity is maximised.
Since the travel is symmetrical, considering the first half ($s=360$) with maximum acceleration and using $s=ut+\frac{1}{2}at^2$:
\begin{align*}
    360&=0+\frac{1}{2}\times1.3\times t^2\\
    \implies t&=\sqrt{\frac{720}{1.3}}
\end{align*}
Since this was only half the distance, the total minimum travel time is $2t$:
$$
    2t=2\sqrt{\frac{720}{1.3}}\approx\boxed{47.1\unit{s}}
$$

\question{A1}{25}{18}
\subquestion{a}
Since we know the distance, initial velocity (0), and time, we can derive the acceleration through the SUVAT equaition $s=ut+\frac{1}{2}at^2$:
\begin{align*}
    12&=0+0.5a\times4^2\\
    \implies a&=\frac{12}{0.5\times4^2}=\frac{12}{8}=1.5\unit{m\,s^{-2}}
\end{align*}
Then using $v=u+at$ we find the velocity at $t=2$ is:
$$
    v=0+1.5\times2=3\unit{m\,s^{-1}}
$$
The answer matching this is \answer{D}.

\subquestion{b}
\graphquestion

\question{A1}{34}{19}
If both projectiles reach the same height, that means that the vertical component of the initial velocity was the same, since:
\begin{align*}
    {u_1}_yt-\frac{1}{2}gt^2&=\max y={u_2}_yt-\frac{1}{2}gt^2\\
    \implies{u_1}_y&={u_2}_y
\end{align*}
If the vertical component of the initial velocity is the same, then the time taken to reach the ground is the same.
This is because both are experiencing the same gravitational pull.
Thus the answer is \answer{B}.

\question{A1}{34}{20}
The horizontal distance travelled is proportional to the height $s\propto h$ since the time taken to hit the ground is proportional to the height $t_\text{final}\propto h$.
The time taken to hit the ground $s_y=-h$, with 0 initial vertical velocity, can be computed as:
\begin{align*}
    -h&=0-\frac{1}{2}gt_\text{final}^2\\
    \implies t_\text{final}&=\sqrt{\frac{2h}{g}}
\end{align*}
Note that the negative solution was discarded as it is physically impossible.
Thus we derive that:
$$
    t_\text{final}\propto\sqrt{h}
$$
Then, we compute that horizontal distance travelled (with initial horizontal velocity $u$, with no or negligible air resistance) as:
$$
    s=ut+0
$$
Therefore, the final distance travelled is $s$ evaluated at $t_\text{final}$, meaning:
$$
    s\propto\sqrt{h}
$$
Thus to reach a distance of $2s$, we need a height of $4h$, since $\sqrt{4h}=2\sqrt{h}$.
This means the answer is \answer{D}.

\question{A1}{34}{21}
Using $s=ut+\frac{1}{2}at^2$:
\begin{align*}
    s&=-4t-\frac{1}{2}gt^2\\
    &=-4\times1.9-\frac{g}{2}(1.9)^2\approx-25.3\unit{m}
\end{align*}
This is the displacement of the object over that time, so the tower's height is $h=-s=25.3\unit{m}$.
The closest answer is therefore \answer{D}.

\question{A1}{34}{22}
The time to hit the floor is given by, as derived in \textbf{A1:P34 Question 20}:
$$
    t_\text{final}=\sqrt{\frac{2h}{g}}
$$
The horizontal displacement is then given through $s=ut+\frac{1}{2}at^2$ (noting air resistance is negligible, providing 0 acceleration):
\begin{align*}
    s&=vt_\text{final}\\
    &=v\sqrt{\frac{2h}{g}}
\end{align*}
The answer is therefore \answer{C}.

\question{A1}{34}{23}
\subquestion{a}
The maximum height is reached when the vertical velocity is 0, since the changing of a sign changes the direction of motion, making the height of the ball strictly decrease post that point.
Therefore, using $v=u+at$:
\begin{align*}
    0&=u_y-gt_\text{max}\\
    u_y&=gt_\text{max}=0.9g\approx\boxed{8.84\unit{m\,s^{-1}}}
\end{align*}

\subquestion{b}
Noting there is no air resistance, the horizontal velocity can be computed using $s=ut+\frac{1}{2}at^2$ as:
\begin{align*}
    16&=0.9u_x+0\\
    \implies u_x&=\frac{16}{0.9}\approx17.8\unit{m\,s^{-1}}
\end{align*}
Using Pythagoras theorem, we get that the total initial velocity is:
$$
    u=\sqrt{u_x^2+u_y^2}=\sqrt{17.8^2+8.84^2}\approx\boxed{19.9\unit{m\,s^{-1}}}
$$

\subquestion{c}
Solving the equation $u\sin\theta=u_y$ (or $u\cos\theta=u_x$) gives:
\begin{align*}
    \sin\theta&\approx\frac{8.84}{19.9}\\
    \implies\theta&\approx\sin^{-1}\frac{8.84}{19.9}\approx0.4603\hdots\unit{rad}
\end{align*}
Converting to degrees (using $\theta_\text{deg}=\frac{180\theta_\text{rad}}{\pi}$):
$$
    \theta\approx\boxed{26.4^ \circ}
$$

\subquestion{d}
Using $s=ut+\frac{1}{2}at^2$:
\begin{align*}
    \max h&=u_yt_\text{max}-\frac{1}{2}gt_\text{max}^2\\
    &=8.84\times0.9-\frac{g}{2}0.9^2\approx\boxed{3.98\unit{m}}
\end{align*}

\question{A1}{34}{24}
Since there is 0 initial vertical velocity, the time to reach the net is (using $s=ut+\frac{1}{2}at^2$):
\begin{align*}
    0.9-2.7&=0-\frac{1}{2}gt_\text{final}^2\\
    \implies t_\text{final}&=\sqrt{\frac{2\times1.8}{g}}\qquad\text{(Negative solution not possible)}\\
    &\approx0.605\unit{s}
\end{align*}
Then to derive the initial horizontal velocity (with assumed 0 acceleration), we use $s=ut+\frac{1}{2}at^2$ again:
\begin{align*}
    12&=u_xt_\text{final}+0\\
    \implies u_x&\approx\frac{12}{0.605}\approx\boxed{19.8\unit{m\,s^{-1}}}
\end{align*}

\question{A1}{34}{25}
\subquestion{a}
The initial vertical velocity $u\sin\theta$ is approximately $0.628\unit{m\,s^{-1}}$.
Then using $s=ut+\frac{1}{2}at^2$ and checking for approximate equality at $t=0.3\unit{s}$:
\begin{align*}
    -0.25&\approx0.628\times0.3-\frac{1}{2}g(0.3)^2\\
    -0.25&\approx-0.2535\quad\color{Green}\checkmark
\end{align*}

\subquestion{b}
\subsubquestion{i}
The initial horizontal velocity $u\cos\theta$ is approximately $8.98\unit{m\,s^{-1}}$.
Using $s=ut+\frac{1}{2}at^2$ with assumed 0 acceleration:
$$
    s\approx8.98\times0.3+0=\boxed{2.69\unit{m}}
$$

\subsubquestion{ii}
Only the vertical velocity is changing (due to the assumed 0 horizontal acceleration).
Therefore, using $v=u+at$:
$$
    v_\text{final}\approx0.628-0.3g=-2.32\unit{m\,s^{-1}}
$$
Then using Pythagoras theorem to calculate the total final velocity:
$$
    v=\sqrt{v_x^2+v_y^2}=\sqrt{8.98^2+(-2.32)^2}\approx\boxed{9.27\unit{m\,s^{-1}}}
$$

\subquestion{c}
\subsubquestion{i}
\graphquestion

\subsubquestion{ii}
\graphquestion

\subsection{Subtopic A.2 – Forces and momentum}
\question{A2}{46}{1}
Disregarding the initial moment, in which force is applied to propell the object, the only force acting on the object is gravity pull.
Gravitational pull is always directed downwards, meaning the answer is \answer{D}.
Note that what is 0 isn't the force, but the velocity (since the sign is changing).

\question{A2}{46}{2}
Ignore all vertical movement.
Using $v^2=u^2+2as$ (with $v=0$ since the pellet comes to rest):
\begin{align*}
    0&=200^2+2a\times0.1\\
    \implies a&=-\frac{40000}{0.2}=-200000\unit{m\,s^{-2}}
\end{align*}
Then using Newton's second law ($\force\newton{2}ma$), and only considering the magnitude (ignoring the $-$ sign):
$$
    \force=0.002\times200000=400\unit{N}
$$
Thus, the answer is \answer{C}.

\question{A2}{47}{3}
The average acceleration is computed using $v=u+at$ as:
\begin{align*}
    15&=0+0.01a\\
    \implies a&=\frac{15}{0.01}=1500\unit{m\,s^{-2}}
\end{align*}
Then using Newton's second law ($\force\newton{2}ma$):
$$
    \force=0.058\times1500=\boxed{87\unit{N}}
$$

\question{A2}{47}{4}
\subquestion{a}
First convert the speed units to $\unit{m\,s^{-1}}$ by dividing through by 3.6:
$$
    \left\{\begin{matrix}
        v=12.5\unit{m\,s^{-1}}\\
        u=22.\bar{2}\unit{m\,s^{-1}}
    \end{matrix}\right.
$$
Then using $v^2=u^2+2as$:
\begin{align*}
    12.5^2&=22.\bar{2}^2+36a\\
    \implies a&=\frac{12.5^2-22.\bar{2}^2}{36}\approx-9.35\unit{m\,s^{-2}}
\end{align*}
Therefore, the average force (only considering magnitude) must be:
$$
    \force\newton{2}ma\approx1200\times9.35\approx\boxed{11220\unit{N}}
$$

\subquestion{b}
Using $v=u+at$:
\begin{align*}
    12.5&\approx22.\bar{2}-9.35t\\
    \implies t&\approx\frac{12.5-22.\bar{2}}{-9.35}\approx\boxed{1.04\unit{s}}
\end{align*}

\question{A2}{47}{5}
\subquestion{a}
Noting that Newton's second law states $\force\newton{2}ma$, we see that:
\begin{align*}
    a=\frac{\force}{m}=\frac{400}{11000}=\frac{2}{55}\unit{m\,s^{-2}}
\end{align*}
Then using $v=u+at$ (and noting that since it was stationary at $t=0$, $u=0$):
$$
    v=0+\frac{2}{55}\times8\approx\boxed{0.291\unit{m\,s^{-1}}}
$$

\subquestion{b}
Using $s=ut+\frac{1}{2}at^2$:
\begin{align*}
    s&=0+\frac{1}{2}\frac{2}{55}\times8^2\\
    &=\frac{64}{55}\approx\boxed{1.16\unit{m}}
\end{align*}

\question{A2}{47}{6}
\subquestion{a}
The force acting on the electron is only acting along the vertical.
Thus, if there is no force acting horizontally, then there is no acceleration horizontally.
Without acceleration, the velocity remains constant (Newton's first law).

\subquestion{b}
\subsubquestion{i}
The time taken for the electron to travel the 25$\unit{cm}$ is given by:
$$
    t=\frac{0.25\unit{m}}{8\times10^6\unit{m\,s^{-1}}}=3.125\times10^{-8}
$$
We then calculate the acceleration acting on the electron using Newton's second law:
$$
    a\newton{2}\frac{\force}{m}=\frac{6.4\times10^{-17}}{9.11\times10^{-31}}=\frac{6.4}{9.11}\times10^{14}\unit{m\,s^{-2}}
$$
Finally using $s=ut+\frac{1}{2}at^2$:
\begin{align*}
    s&=0+\frac{1}{2}\frac{6.4}{9.11}\times10^{14}\times\qty(3.125\times10^{-8})^2\\
    &\approx\boxed{3.43\unit{cm}}
\end{align*}

\subsubquestion{ii}
Using $v=u+at$ we get a vertical velocity of:
\begin{align*}
    v_y&=0+\frac{6.4}{9.11}\times10^{14}\times3.125\times10^{-8}\\
    &\approx2.2\times10^6\unit{m\,s^{-1}}
\end{align*}
Then using $\displaystyle\tan\theta=\frac{v_y}{v_x}$ ($v_x$ is the velocity stated in the question):
\begin{align*}
    \theta=\tan^{-1}\frac{2.2\times10^6}{8\times10^6}\approx\boxed{15.4^\circ}
\end{align*}

\question{A2}{47}{7}
\subquestion{a}
\graphquestion
If there is no acceleration, then no force other than gravity is acting on the person.
Thus the scale, measuring force, will show just the weight force:
$$
    \force_g\newton{2}mg=75\times9.82\approx\boxed{737\unit{N}}
$$

\subquestion{b}
\graphquestion
The force driving the elevator upwards will push against the person in the elevator.
By Newton's third law, the person will then experience an equal force, but in the oposite direction.
Since the force on the elevator acts upwards, the force on the person acts downwards with an acceleration of $2\unit{m\,s^{-2}}$.

The magnitude of the force acting downwards on the person is given as:
$$
    \force_\downarrow\newton{2}ma=75\times2=150\unit{N}
$$

Therefore the magnitude of the resultant force $\force_R$ is given as ($\force_g$ was computed in \subquestion{a}):
$$
    \force_R=\force_g+\force_\downarrow\approx737+150=\boxed{887\unit{N}}
$$

\question{A2}{54}{8}
\subquestion{a}
Let the force of tension in each string be denoted $\force_t$.
Combined, the vertical component of both string's tension ($\force_t\sin\theta$) force, $2\force_t\sin\theta$, must balance the weight force of the object, $\force_g$.
The weight force of the object is given as:
$$
    \force_g\newton{2}mg=2g\approx19.6\unit{N}
$$
Then to derive the angle $\theta$, splitting the 150$^\circ$ in to two gives 75$^\circ$.
Then, in consideration of the right angle triangle on the side towards the weight, we get the angle $\theta=90-75=15^\circ$.
Finally, we solve the equation:
\begin{align*}
    2\force_t\sin15^\circ&=\force_g\\
    \force_t&=\frac{19.6}{2\sin15^\circ}\approx\boxed{37.9\unit{N}}
\end{align*}

\subquestion{b}
The threads can only support a certain amount of tension.
As the angle (let this be called $\phi$) between the strings increases, the angle $\theta$ decreases.
This is because (as demonstrated in \subquestion{a}):
$$
    \theta=90-\frac{\phi}{2}
$$
The equation for the force of tension is proportional to (as demonstrated in \subquestion{a}):
\begin{align*}
    \force_t&\propto\frac{1}{\sin\theta}=\frac{1}{\sin90-\frac{\phi}{2}}\\
    &=\frac{1}{\cos\frac{\phi}{2}}\\
    \therefore\force_t&\propto\frac{1}{\cos\phi}
\end{align*}
As $\phi$ increases towards 180$^\circ$, $\cos\phi$ increases as well, meaning the force of tension, too, increases.
Thus be increasing the angle, the threads are more likely to break.

\question{A2}{54}{9}
\subquestion{a}
\graphquestion

\subquestion{b}
The tension in each string has to support the weight force pulling on it.
For the upper string, both A ($m_A=M$) and B ($m_B=2M$) pull on it.
Therefore:
\begin{align*}
    {\force_t}_\text{up}&={\force_g}_A+{\force_g}_B\\
    &\newton{2}Mg+2Mg=\boxed{3Mg}
\end{align*}
For the second string, only the weight of $B$ is pulling on it, thus:
$$
    {\force_t}_\text{down}={\force_g}_B=\boxed{2Mg}
$$

\question{A2}{54}{10}
\subquestion{a}
\subsubquestion{i}
To form a 40$^\circ$ angle, the angle formed between the bob, the vertical and the end point of $\force$ if placed at the end of the weight force must be 40$^\circ$.
We therefore solve the equaition:
\begin{align*}
    \tan40^\circ&=\frac{\force}{0.5}\\
    \implies\force&\approx0.839\times0.5\approx\boxed{0.420\unit{N}}
\end{align*}

\subsubquestion{ii}
To stay in equilibrium and not break, the magnitude of the tension force in the string must be equal to the magnitude of the resultant.
We thus calculate the magnitude of the resultant, using Pythagoras' theorem, as:
$$
    \force_R\approx\sqrt{0.5^2+0.420^2}=\boxed{0.653\unit{N}}
$$

\subquestion{b}
Since the forces remain unchanged, but there is no tension to hold them back anymore, the bob starts accelerating along the resultant.
It thus moves away from the wall at the same direction the equilibrium was in (40$^\circ$ away from the negative vertical).

\question{A2}{59}{11}
When a spring is cut in half, the spring constant doubles, thus $k'=2k$.
Then, when placed in parallel, the total spring constant is the sum of the individual spring constants, thus:
$$
    k_\text{total}=k'+k'=4k
$$
Using Hooke's law, we know that since the object is in equilibrium:
\begin{align*}
    \force_g\newton{2}mg=\force_h&=-kL\\
    \implies mg&=-kL
\end{align*}
Applying the new total spring constant:
$$
    mg=-k'L'=-4kL'
$$
Since both equations equal $mg$:
\begin{align*}
    -kL&=-4kL'\\
    \implies L'&=\frac{L}{4}
\end{align*}
The answer is thus \answer{A}.
We also conclude that for some spring constant $k'=nk$, the extension of the new system is $L'=\frac{L}{n}$.

\question{A2}{59}{12}
The spring constant of a system connected in series is given by:
\begin{align*}
    \frac{1}{k'}&=\frac{1}{k_1}+\frac{1}{k_2}\\
    &=\frac{1}{k}+\frac{1}{2k}=\frac{3}{2k}\\
    \therefore k'&=\frac{2k}{3}
\end{align*}
We define using Newton's second law and Hooke's law the total extension $L$ by:
\begin{align*}
    \force_g\newton{2}mg=\force_h&=-k'L=-\frac{2k}{3}L\\
    \implies mg&=-\frac{2k}{3}L
\end{align*}
The extension of the first spring is similarly given as:
$$
    mg=-kL_1
$$
Equating both expressions:
\begin{align*}
    -kL_1&=-\frac{2k}{3}L\\
    \implies L_1&=\frac{2L}{3}
\end{align*}
Thus the answer fitting the total spring constant and extension is \answer{D}.

\question{A2}{61}{13}
Let the density of object 1 and 2 be denoted $\rho_1$ and $\rho_2$, and the density of water $\rho_w$.
Dividing both sides of the ratio $\frac{\rho_1}{\rho_2}$ by $\rho_w$ gives:
$$
    \frac{\rho_1}{\rho_2}\stackrel{\text{let}}{=}r=\frac{\frac{\rho_1}{\rho_w}}{\frac{\rho_2}{\rho_w}}
$$
The fraction of an object's volume which is below water is given as:
$$
    V'=\frac{\rho_o}{\rho_w}
$$
We know $75\%$ of the volume of object 1 is below the water, and $50\%$ for object 2.
Thus:
$$
    r=\frac{75\%}{50\%}=1.5
$$
This makes the answer \answer{A}.

\question{A2}{61}{14}
\subquestion{a}
The percentage of the volume of the cuboid under the water is the ratio between the submerged height and the total height.
Therefore the percentage is given by:
$$
    \frac{2\unit{cm}}{10\unit{cm}}=20\%
$$
We then get the density of object using the fact that the ratio between an object's density and the fluid's density is the percentage of its submerged volume:
$$
    0.2=\frac{\rho_o}{\rho_w}\implies\rho_o=0.2\rho_w
$$
Finally, we calculate mass using $m=V\rho$:
\begin{align*}
    m&=0.25\times0.25\times0.1\times0.2\rho_w\\
    &=0.00125\rho_w=0.00125\times10^3=\boxed{1.25\unit{kg}}
\end{align*}

\subquestion{b}
To sink, the whole $10\unit{cm}$ has to submerged.
The new mass is given as:
$$
    m=1.25+4=5.25\unit{kg}
$$
We then need to calculate the new density of the cuboid, which is done as (using the volume computed in \subquestion{a}):
\begin{align*}
    \rho_o=\frac{m}{V}=\frac{5.25}{0.25\times0.25\times0.1}=\frac{5.25}{0.00625}=840\unit{kg\,m^{-3}}
\end{align*}
Finally, we compute the ratio of the object's density and the water's density:
$$
    \frac{\rho_o}{\rho_w}=\frac{840}{1000}\leq1
$$
Since the ratio is less than 1, all of the volume will not be submerged, making the object stay afloat.

\question{A2}{62}{15}
To float in equilibrium, the density of the balloon + the density of the gas have to be equal to the density of air.
Therefore:
$$
    \rho_a=\rho_b+\rho_g
$$
The density of the balloon $\rho_b$ (sans gas) is computed as:
$$
    \rho_b=\frac{m}{V}=\frac{0.014}{\frac{4}{3}\pi0.15^3}\approx0.990\unit{kg\,m^{-3}}
$$
Since the density of air $\rho_a$ was given, we simply solve the original equation:
$$
    1.2=0.99+\rho_g\implies\rho=1.2-0.99=\boxed{0.21\unit{kg\,m^{-3}}}
$$

\question{A2}{62}{16}
\subquestion{a}
Using $\force_a=\rho gV$ (the density of air $\rho_a$ was given as $1.2\unit{kg\,m^{-3}}$):
\begin{align*}
    \force_a&=1.2g\times\frac{4}{3}\pi\times0.03^3\\
    &\approx\boxed{1.33\times10^{-3}\unit{N}}
\end{align*}

\subquestion{b}
Now the force of buoyancy does not support the ball's weight force, meaning the tension will have to make up for it.
Previously:
\begin{align*}
    {\force_t}_\text{before}&=\force_g-\force_t\\
    &\newton{2}0.0018\times9.82-1.33\times10^{-3}\approx0.0163\unit{N}
\end{align*}
Now:
$$
    {\force_t}_\text{after}=\force_g\newton{2}0.0018\times9.82=0.0177\unit{N}
$$
Thus the percentage multiplier is given by the ratio $\frac{{\force_t}_\text{after}}{{\force_t}_\text{before}}$:
$$
    \frac{{\force_t}_\text{after}}{{\force_t}_\text{before}}=\frac{0.0177}{0.0163}\approx1.0859
$$
Therefore the percentage change is $1-\text{the multiplier}$:
$$
    1-1.0859=\boxed{8.59\%}
$$

\question{A2}{67}{17}
\subquestion{a}
First conver the speed $50\unit{km\,h^{-1}}$ to $\unit{m\,s^{-1}}$ by dividing by 3.6:
$$
    u=\frac{50}{3.6}\approx13.9\unit{m\,s^{-1}}
$$
Using $v^2=u^2+2as$:
\begin{align*}
    0&=13.9^2+50a\\
    \implies a&\approx-\frac{193}{50}=\boxed{-3.86\unit{m\,s^{-2}}}
\end{align*}

\subquestion{b}
The maximum force the static friction can handle is given as:
$$
    \max\force_{\mu_s}=\mu_s\force_N
$$
In this case, $\force_N=\force_g$.
Therefore:
$$
    \max\force_{\mu_s}\newton{2}0.45mg\approx4.419m\unit{N}
$$
The acceleration of the truck will apply force to the box, which is computed using Newton's second law as:
$$
    \force_\text{truck}\newton{2}ma\approx3.86m\unit{N}
$$
Thus, we see that the force applied by the truck $\force_\text{truck}<\max\force_{\mu_s}$, meaning the static friction will keep the box stationary.

\question{A2}{67}{18}
The force applied by the dynamic friction opposing the motion is given by:
$$
    \force_{\mu_d}=\mu_d\force_N
$$
In this case, $\force_N=\force_g\newton{2}mg\approx19.6\unit{N}$.
Consequently, the acceleration caused by the friction (using Newton's second law) is:
$$
    a=\frac{\force_{\mu_d}}{m}=\frac{0.1\times19.6}{2}=0.982\unit{m\,s^{-2}}
$$
Then, using $v^2=u^2+2as$:
\begin{align*}
    v&=\sqrt{8^2-2\times0.982\times16}\\
    &\approx\boxed{5.71\unit{m\,s^{-1}}}
\end{align*}

\question{A2}{67}{19}
\subquestion{a}
Using $s=ut+\frac{1}{2}at^2$:
\begin{align*}
    0.5&=0+\frac{1}{2}a\times3.8^2\\
    a&=\frac{1}{3.8^2}\approx\boxed{0.0693\unit{m\,s^{-2}}}
\end{align*}

\subquestion{b}
The angle the gravitational force (straight down) makes with the ramps incline is 15$^\circ$.
Thus the component of the weight of the box parallel to the ramp is:
\begin{align*}
    {\force_g}_\text{ramp}&=\force_g\sin15^\circ\\
    &\newton{2}2\times9.82\times\sin15^\circ\\
    &\approx\boxed{5.08\unit{N}}
\end{align*}

\subquestion{c}
The resultant force experienced by the box is given using Newton's second law as:
$$
    \force_R\newton{2}ma=2\times0.0693=0.139\unit{N}
$$
The two forces working in the direction of motion is the force of friction, the force of gravity parallel to the ramp, and the pull force, meaning:
\begin{align*}
    \force_R&=8-{\force_g}_\text{ramp}-\force_\mu\\
    \implies\force_\mu&=8-{\force_g}_\text{ramp}-\force_R=8-5.08-0.139\approx\boxed{2.78\unit{N}}
\end{align*}

\subquestion{d}
The normal force is the component of gravity perpendicular to the ramp, which is given as:
\begin{align*}
    \force_N&=\force_g\cos15^\circ\\
    &\newton{2}2\times9.82\times\cos15^\circ19.0\unit{N}
\end{align*}
Finally, since frictional force is defined as $\force_\mu=\mu_d\force_N$:
\begin{align*}
    2.78&=\mu_d19.0\\
    \therefore\mu_d&=\frac{2.78}{19.0}\approx\boxed{0.146}
\end{align*}

\question{A2}{67}{20}
\subquestion{a}
For the static friction to support the book, $\mu_s\force_N\geq\force_g$.
We choose the minimum ($\mu_s\force_N=\force_g$).
The normal force is in this case the force $\force$ being applied to the book against the wall (since the normal force is the wall pushing back on the book).
Since the book's weight ($\force_g$) and $\mu_s$ were given, we simply solve the equation:
\begin{align*}
    0.75\force&=12\\
    \implies\force&=\frac{12}{0.75}=\boxed{16\unit{N}}
\end{align*}

\subquestion{b}
\graphquestion

\subquestion{c}
The force of dynamic friction opposing the motion is given as:
\begin{align*}
    \force_\mu&=\mu_d\force_N\\
    &=0.6\times10=6\unit{N}
\end{align*}
The resultant force $\force_R$ is therefore:
\begin{align*}
    \force_R&=\force_g-\force_\mu\\
    &=12-6=6\unit{N}
\end{align*}
This is half of the original weight force, meaning that since the mass is constant, the acceleration is $\frac{g}{2}$.
This is approximately:
$$
    \frac{g}{2}=\boxed{4.91\unit{m\,s^{-2}}}
$$

\question{A2}{72}{21}
Since the skydiver cannot magically start falling upwards because of the parachute, the direction of the velcity remains constant.
At $t_1$ however, the speed was increasing, meaning there was positive acceleration.
At $t_2$, the oposite is true, meaning there was negative acceleration.
Thus, the answer is $\answer{B}$.

\question{A2}{72}{22}
The terminal velocity is given by the formula:
$$
    v_t=\frac{(\rho_s-\rho_f)gV}{6\pi\eta r}
$$
Since $V=\frac{4}{3}\pi r^3$,
$$
    v_t\propto\frac{r^3}{r}\implies v_t\propto r^2
$$
Therefore, if the radius is 2 times larger, then the terminal velocity is $2^2=4$ times larger, making the answer \answer{C}.

\question{A2}{72}{23}
There are two forces acting on the ball, buoyancy $\force_a$ and drag force $\force_\delta$.
The buoyancy force is given by Archimede's principle as:
$$
    \force_a=\rho_fgV
$$
where $\rho_f$ is the density of the fluid.
Since $m=\rho_sV$ and $\rho_s=1.5\rho_f$:
\begin{align*}
    \force_a&=\rho_fg\frac{m}{1.5\rho_f}=\frac{mg}{1.5}\newton{2}\frac{\force_g}{1.5}
\end{align*}
The resultant force is zero (since there is no acceleration), and is given by:
\begin{align*}
    \force_R=0&=\force_g-\force_a-\force_\delta\\
    \implies\force_\delta&=\force_a-\force_g\\
    &=\frac{\force_g}{1.5}-\force_g\\
    &=\frac{1.2}{1.5}-1.2=-0.4\unit{N}
\end{align*}
The magnitude of the drag force is therefore $0.4\unit{N}$, making the answer \answer{A}.

\question{A2}{72}{24}
\subquestion{a}
Drag force $\force_\delta$ is proportional to speed whilst weight force and buoyancy are constant.
Therefore, the drag force keeps increasing as the ball accelerates downwards until it reaches a velocity (the terminal velocity), in which all of the forces balance, and the ball experiences net zero force.
This net zero means 0 acceleration, keeping the velocity at this point.

\subquestion{b}
\subsubquestion{i}
Since $m=\rho V$ and $V=\frac{4}{3}\pi r^3$,
\begin{align*}
    m&=8000\times\frac{4}{3}\pi\times0.002^3\\
    &\approx0.000268\unit{kg}
\end{align*}
The weight is then given by Newton's second law as:
$$
    \force_g\newton{2}mg\approx0.000268\times9.82\approx\boxed{2.63\times10^{-3}\unit{N}}
$$

\subsubquestion{ii}
By Archimede's principle:
\begin{align*}
    \force_a&=\rho gV\\
    &=920\times9.82\times\frac{4}{3}\pi\times0.002^3\\
    &\approx\boxed{3.03\times10^{-4}\unit{N}}
\end{align*}

\subquestion{c}
Using Stoke's law:
\begin{align*}
    v_t&=\frac{(\rho_s-\rho_f)gV}{6\pi\eta r}\\
    &=\frac{(8000-920)\times9.82\times\frac{4}{3}\pi\times0.002^3}{6\pi\times8.4\times10^{-2}\times0.002}\\
    &\approx\boxed{0.736\unit{m\,s^{-1}}}
\end{align*}

\question{A2}{75}{25}
\subquestion{a}
Using $v=u+at$ (with $v$ being in the oposite direction of $u$, thus negative):
\begin{align*}
    -6&=9+0.05a\\
    \implies a&=\frac{-15}{0.05}=\boxed{-300\unit{m\,s^{-2}}}
\end{align*}

\subquestion{b}
By Newton's second law (considering only magnitude):
$$
    \force\newton{2}ma=0.4\times300=\boxed{120\unit{N}}
$$

\question{A2}{75}{26}
\subquestion{a}
\subsubquestion{i}
The pellet looses all momentum, which means the change in momentum is:
$$
    \Delta p=mv=0.002\times180=\boxed{0.36\unit{N\,s}}
$$

\subsubquestion{ii}
The acceleration is computed using Newton's second law as:
$$
    a\newton{2}\frac{\force}{m}=\frac{750}{0.002}=375000\unit{m\,s^{-2}}
$$
Then using $v=u+at$:
$$
    0=180-375000t\implies t=\frac{180}{375000}=\boxed{4.8\times10^4\unit{s}}
$$

\subquestion{b}
Using $v^2=u^2+2as$:
\begin{align*}
    0&=180^2-2\times375000s\\
    \implies s&=\frac{180^2}{750000}=\boxed{0.0432\unit{m}}
\end{align*}

\subquestion{c}
Because it assumes constant acceleration (and force) from the average force.
This is not realistic, and the acceleration likely varies over time as the pellet penetrates the block.

\question{A2}{77}{27}
The area under the graph is the change in momentum.
Therefore:
\begin{align*}
    \Delta p&=\frac{(1200-0)(3.5-2)\times10^{-3}}{2}+\frac{(1200-0)(7-3.5)\times10^{-3}}{2}\\
    &=3\unit{N\,s}
\end{align*}
Since $p=mv$, the change in velocity is thus:
$$
    \Delta v=\frac{\Delta p}{m}=\frac{3}{0.15}=20\unit{m\,s^{-1}}
$$
Thus (noting that since it's a change in direction, we let the original 8$\unit{m\,s^{-1}}$ be negative):
$$
    v_\text{post hit}=20-8=12\unit{m\,s^{-1}}
$$
Consequently, the answer is \answer{B}.

\question{A2}{77}{28}
Let $x$ be the height of one unit square.
Each square has a width of $10^{-2}$
The change in momentum is (very approximately by counting rough squares) then:
$$
    \Delta p=10\times(x\times10^{-2})=\frac{x}{10}
$$
Since the change in momentum was 10$\unit{N\,s}$, we let $x\approx100$.
Finally, because $\force_\text{max}$ is 5 units up, $\force_\text{max}\approx5x=500\unit{N}$.
The closest answer is therefore \answer{D}.

\question{A2}{77}{29}
\subquestion{a}
\subsubquestion{i}
The change in momentum between $t=0$ and $0.5\unit{s}$ is $0.75\unit{N\,s}$.
The only force acting to change this is the weight force.
Therefore:
$$
    \force_g=\frac{\Delta p}{t}=\frac{0.75}{0.5}=\boxed{1.5\unit{N}}
$$

\subsubquestion{ii}
Using $v=u+at$:
$$
    v=0+0.5g\approx\boxed{4.91\unit{m\,s^{-1}}}
$$

\subquestion{b}
The change in momentum during the contact (between $t=0.5$ and $0.6\unit{s}$) is $0.5-(-0.75)=1.25\unit{N\,s}$.
Dividing this by the contact time of $0.1\unit{s}$ gives us the average force:
$$
    \bar{\force}=\frac{1.25}{0.1}=\boxed{12.5\unit{N}}
$$

\question{A2}{79}{30}
\subquestion{a}
\subsubquestion{i}
The thrust force is given by $\force_\tau=\dot{m}v_e$ where $\displaystyle\dot{m}=\dv{m}{t}$ is the rate of change of the mass expulsion and $v_e$ the velocity of the explused material.
Therefore:
$$
    \force_\tau=2.8\times3.6\times10^3\approx\boxed{10000\unit{N}}
$$

\subsubquestion{ii}
Using Newton's second law:
$$
    a\newton{2}\frac{\force_\tau}{m}=\frac{10000}{4\times10^4}=\boxed{0.25\unit{m\,s^{-2}}}
$$

\subquestion{b}
The thrust force remains constant (since $\dot{m}$ and $v_e$ are by design constant), but the mass of the rocket decreases.
By Newton's second law:
$$
    a\stackrel{\text{N2}}{\propto}\frac{1}{m}
$$
If $m$ therefore decreases, the acceleration increases.
When the rocket then stops thrusting, the acceleration becomes $g$.

\subquestion{c}
The final mass is $m=40000-2.8\times25\times60=35800\unit{kg}$.
Therefore the final acceleration is:
$$
    a\newton{2}\frac{\force_\tau}{m}=\frac{10000}{35800}\approx0.279\unit{m\,s^{-1}}
$$
Using $v=\bar{a}t$:
$$
    v=\frac{0.25+0.279}{2}\times25\times60\approx\boxed{397\unit{m\,s^{-1}}}
$$
Note that this answer differs from the one given in the answer sheeet provided by Oxford.
However, there ansewr only seems to use the final acceleration, which would assume constant acceleration across, when the question before states clearly this is not the case.
I therefore decided to keep my answer which I believe to be more correct.

\subquestion{d}
Using $\force_\tau=\dot{m}v_e$:
\begin{align*}
    65000&=3.6\times10^3\times\dot{m}\\
    \implies\dot{m}&=\frac{65000}{3600}\approx\boxed{18.1\unit{kg\,s^{-1}}}
\end{align*}

\question{A2}{85}{31}
\subquestion{a}
The only part which changed is that it went from $-p_y$ to $p_y$ where $p_y$ was the magnitude of the vertical momentum.
The change is thus:
$$
    \Delta p=p_y-(-p_y)=2p_y=2mv\sin\theta\unit{kg\,m\,s^{-1}}
$$
This makes the answer \answer{C}.

\subquestion{b}
Re-itterating what was stated in \subquestion{a}, the only part which changed was the vertical momentum.
It went from pointing down to pointing up.
Thus the vector must point up, making the answe \answer{D}.

\question{A2}{85}{32}
Recall that the momentum must be conserved.
The total momentum (positive direction to the right) before the collision was:
$$
    p=2mv-mv=mv\unit{kg\,m\,s^{-1}}
$$
After the collision, the carts stick together.
We thus think of them as one of object with combined mass $2m$.
Therefore, using the definition $p=mv$:
$$
    mv=2m\implies v=\frac{v}{2}=0.5m\unit{m\,s^{-1}}
$$
This makes the answer \answer{A}.

\question{A2}{85}{33}
\subquestion{a}
The momentum before the explosion:
$$
    p=mv=2\times6=12\unit{kg\,m\,s^{-1}}
$$
Since the small piece stopped, $v=0$, meaning that the larger piece carries all the momentum.
Thus:
$$
    v\stackrel{\Delta}{=}\frac{p}{m}=\frac{12}{1.5}=\boxed{8\unit{m\,s^{-1}}}
$$

\subquestion{b}
Using $E_k=\frac{1}{2}mv^2$ (note that small piece has 0 kinetic energy since $v=0$):
\begin{align*}
    {E_k}_\text{before}&=\frac{1}{2}\times2\times6^2=36\unit{J}\\
    {E_k}_\text{after}&=\frac{1}{2}\times1.5\times8^2=48\unit{J}
\end{align*}
The gain is thus:
$$
    {E_k}_\text{after}-{E_k}_\text{before}=48-36=\boxed{12\unit{J}}
$$

\question{A2}{85}{34}
\subquestion{a}
The momentum after the collision is:
$$
    p_\text{after}=0.002\times150+0.05\times2.4=0.42\unit{kg\,m\,s^{-1}}
$$
Before the collision, the box had no momentum, meaning the pellet had all of it.
Therefore:
$$
    v=\frac{p}{m}=\frac{0.42}{0.002}=\boxed{210\unit{m\,s^{-1}}}
$$

\subquestion{b}
\subsubquestion{i}
Using $v=u+at$:
\begin{align*}
    150&=210+a\times1.5\times10^{-4}\\
    \implies a&=\frac{150-210}{1.5\times10^{-4}}=\frac{-600000}{1.5}=\boxed{-4\times10^5\unit{m\,s^{-2}}}
\end{align*}

\subsubquestion{ii}
Using Newton's second law (and only considering magnitude):
$$
    \force\newton{2}ma=0.002\times4\times10^5=\boxed{800\unit{N}}
$$

\question{A2}{85}{35}
\subquestion{a}
The $2000\unit{kg}$ truck recieved all the momentum post-collision since the other truck stopped, meaning:
$$
    p_\text{after}=2000\times6=1.2\times10^4\unit{kg\,m\,s^{-1}}
$$
The momentum before is given by the equation (considering right to be the positive direction):
$$
    p_\text{before}=6000v-2000v=4000v
$$
Equating both and solving for $v$:
$$
    4000v=1.2\times10^4\implies v=\frac{1.2\times10^4}{4000}=\boxed{3\unit{m\,s^{-1}}}
$$

\subquestion{b}
If the collision is elastic, then all energy is conserved.
Before the collision:
$$
    {E_k}_\text{before}=\frac{1}{2}\times6000\times3^2+\frac{1}{2}\times2000\times3^2=36000\unit{J}
$$
After the collision:
$$
    {E_k}_\text{after}=\frac{1}{2}\times2000\times6^2=36000\unit{J}
$$
Since ${E_k}_\text{before}={E_k}_\text{after}$, it has been shown that the collision was elastic.

\question{A2}{88}{36}
\subquestion{a}
Since momentum is conserved along all directions, we consider the horizontal.
Then the first body contributes 0 momentum, meaning the horizontal component of the body of interest ($p\cos\theta$) must be equal to the original momentum.
$$
    1.6\times1=1.6=2v\cos\theta
$$
For the vertical, the vertical component of the momentum of the body of interest must be equal to, but oposite the momentum of the first body:
$$
    0.8\times1=0.8=2v\sin\theta
$$
Squaring both equations and adding both equations together (and using the identity $\cos^2\theta+\sin^2\theta=1$):
\begin{align*}
    1.6^2+0.8^2&=4v^2\cos^2\theta+4v^2\sin^2\theta\\
    \implies3.2&=4v^2\\
    \therefore v&=\pm\sqrt{\frac{3.2}{4}}\\
    \leadsto v&=\boxed{0.894\unit{m\,s^{-1}}}\qquad\text{(since we only consider the mangitude)}
\end{align*}
Then, we return to either of the original equations.
We choose the vertical.
\begin{align*}
    0.8&\approx2\times0.894\sin\theta\\
    \implies\theta&=\sin^{-1}\frac{0.4}{0.894}\approx\boxed{26.6^\circ}
\end{align*}

\subquestion{b}
For the collision to be elastic, the kinetic energy before must equal the kinetic energy post-collision.
For the before energy:
$$
    {E_k}_\text{before}=\frac{1}{2}\times1\times1.6^2=1.28\unit{J}
$$
For the after energy:
\begin{align*}
    {E_k}_\text{after}&\approx\frac{1}{2}\times1\times0.8^2+\frac{1}{2}\times2\times0.894^2\\
    &\approx0.32+0.799=1.12\unit{J}
\end{align*}
We thus conclude the kinetic energies are not equal before and after the collision (and that it can't be a rounding error), meaning the collision was not elastic.

\question{A2}{88}{37}
\subquestion{a}
The kinetic energy before the collision was:
$$
    {E_k}_\text{before}=\frac{1}{2}\times m\times200^2=20000m
$$
The kinetic energy after the collision was:
$$
    {E_k}_\text{after}=\frac{1}{2}\times m\times160^2+\frac{1}{2}\times4m\times v^2
$$
Equating the two since energy is conserved:
\begin{align*}
    {E_k}_\text{before}&={E_k}_\text{after}\\
    \implies20000m&=12800m+2mv^2\\
    \implies v&=\sqrt{\frac{7200}{2}}=\boxed{60\unit{m\,s^{-1}}}
\end{align*}

\subquestion{b}
The vertical component of the momentum of the red body must be equal to, but opposing, the vertical component of the blue body, since the original momentum has no vertical component.
Thus:
\begin{align*}
    160m\sin82.8^\circ&=60\times4m\times\sin\theta\\
    \implies\theta&=\sin^{-1}\left(\frac{160}{60\times4}\times\sin82.8^\circ\right)\\
    &\approx\boxed{41.4^\circ}
\end{align*}

\question{A2}{90}{38}
\subquestion{a}
First we convert the volume flow rate of $9\unit{L\,\text{minute}^{-1}}$ to cubic metres per second as as $1.5\times10^{-4}\unit{m^3\,s^{-1}}$.
The mass flow rate is then the volume flow rate times the density of water:
$$
    \dot{m}=1.5\times10^{-4}\times1000=0.15\unit{kg\,s^{-1}}
$$
Note that the rate of change of mass is equal to the infinitesimal volume of water moving through the hose per unit second times the density of the water.
We calculate this infinitesimal volume as the cross-sectional area times the infinitesimal distance travelled:
$$
    \dv{m}{t}=\rho_w\times A\times\dv{s}{t}
$$
Note that since $v\stackrel{\Delta}{=}\dv{s}{t}$ (noting that $r=\frac{d}{2}$):
\begin{align*}
    v&=\frac{\dot{m}}{A\rho_w}\\
    &=\frac{0.15}{\pi\times0.007^2\times1000}\\
    &\approx\boxed{0.974\unit{m\,s^{-1}}}
\end{align*}

\subquestion{b}
Using the formula derived in \subquestion{a} for the velocity pre-nozzle:
\begin{align*}
    v&=\frac{\dot{m}}{A\rho_w}\\
    &=\frac{0.15}{\frac{\pi\times0.007^2}{12}\times1000}\\
    &\approx11.7\unit{m\,s^{-1}}
\end{align*}
Using the formula $\Delta p=\dot{m}(v-u)$:
\begin{align*}
    \Delta p&=0.15\times(11.7-0.974)\\
    &\approx0.15\times10.7\approx\boxed{1.61\unit{N\,s}}
\end{align*}

\subquestion{c}
Since the change in momentum is non-zero, a force is exerted by the water on the hose.
The force will want to accelerate the hose, so to keep it stationary one must apply a force to the hose.

\question{A2}{92}{39}
\subquestion{a}
The lift force has to oppose the weight force of the weight force, thus the magnitude of the two are equal.
Consequently:
\begin{align*}
    \force_l=\force_g&=3\times10^3\times9.82\\
    &\approx\boxed{29500\unit{N}}
\end{align*}

\subquestion{b}
The lift force exerted is the rate of change of the momentum of the air:
$$
    \force_l=\dot{p}=v\dot{m}=v\times A\rho\times\dv{s}{t}
$$
Therefore, since $p=mv$ (and since $\displaystyle v=\dv{s}{t}$):
$$
    \boxed{\dot{m}=\frac{\dot{p}}{v}=\rho Av}
$$

\subquestion{c}
Using the relationship stated in \subquestion{a}:
\begin{align*}
    \force_l=v\dot{m}=\rho Av^2&=\force_g\approx29500\\
    v&\approx\sqrt{\frac{29500}{95\times1.2}}\approx\boxed{16.1\unit{m\,s^{-1}}}
\end{align*}

\subquestion{d}
The acceleration due to lift is, by Newton's second law:
$$
    a\newton{2}\frac{\force_l}{m}=\frac{\rho Av^2}{m}
$$
We now know that the magnitude of the acceleration due to lift must be 1.2 units greater than the downward acceleration due to gravity.
This means:
\begin{align*}
    \frac{\rho Au^2}{m}&=g+1.2\\
    \implies u&\approx\sqrt{\frac{3\times10^3\times(g+1.2)}{1.2\times95}}\approx\boxed{17.0\unit{m\,s}}
\end{align*}

\question{A2}{92}{40}
\subquestion{a}
Using Newton's second law we get that:
$$
    a\newton{2}\frac{\force}{m}=\frac{6000}{64}
$$
Then using $v=u+at$ we can figure out the time needed to produce this worst-case scenario (converting $\unit{hm\,h^{-1}}$ to $\unit{m\,s^{-1}}$ by dividing by 3.6):
$$
    t=\frac{\frac{45}{3.6}\times64}{6000}=\boxed{0.1\bar{3}\unit{s}}
$$

\subquestion{b}
Using $s=ut+\frac{1}{2}at^2$ (noting that the acceleration acts in the oposite direction of the motion):
\begin{align*}
    s&=\frac{45}{3.6}\times0.1\bar{3}-\frac{1}{2}\times\frac{6000}{64}\times0.1\bar{3}^2\\
    &\approx\boxed{0.833\unit{m}}
\end{align*}

\question{A2}{96}{41}
\subquestion{a}
One revolution around the sun is $\theta=2\pi\unit{rad}$, and then since we know one year is $365.25\unit{days}$, we can convert this in to seconds:
$$
    t_\text{orbit}=365.25\times24\times60\times60\approx3.16\times10^7\unit{s}
$$
Therefore the angular velocity is:
$$
    \omega=\frac{\theta}{t_\text{orbit}}=\frac{2\pi}{3.16\times10^7}\approx\boxed{1.99\times10^{-7}\unit{rad\,s^{-1}}}
$$

\subquestion{b}
The circumference of earth's orbit is:
$$
    P=2\pi\times1.5\times10^{11}=3\pi\times10^{11}\unit{m}
$$
Thus, using the $t_\text{orbit}$ determined in \subquestion{a}, the speed is:
$$
    v=\frac{P}{t_\text{orbit}}\approx\frac{3\pi\times10^{11}}{3.16\times10^7}\approx\boxed{2.98\times10^4\unit{m\,s^{-1}}}
$$

\question{A2}{96}{42}
\subquestion{a}
If the blades make 670 revolutions in one minute, then:
$$
    \omega=\frac{670}{60}\approx11.2\unit{Hz}
$$
Knowing that one revolution is $2\pi\unit{rad}$:
$$
    \omega\approx\boxed{70.4\unit{rad\,s^{-1}}}
$$

\subquestion{b}
Using the formula $v_\text{tip}=\omega r$:
$$
    v_\text{tip}\approx70.4\times0.16\approx\boxed{11.3\unit{m\,s^{-1}}}
$$

\question{A2}{97}{43}
Using the formula $a=\omega^2r$, we deduce:
$$
    a\propto r
$$
Therefore, the ratio is computed as (knowing both have the same angular speed):
$$
    \frac{a_P}{a_Q}=\frac{R}{\frac{R}{2}}=2
$$
Thus, the answer is \answer{C}.

\question{A2}{97}{44}
\subquestion{a}
Since the rotation frequency is 5$\unit{Hz}$:
$$
    t=\frac{1}{5}=0.2\unit{s}
$$
The circumference of a neutron star is:
$$
    P=2\pi\times10^4\unit{m}
$$
Thus, the speed is:
$$
    v=\frac{P}{t}=\frac{2\pi\times10^4}{0.2}=\pi\times10^5\approx\boxed{3.14\times10^5\unit{m\,s^{-1}}}
$$

\subquestion{b}
Using the formula $a=\frac{v^2}{r}$:
$$
    a\approx\frac{\qty(3.14\times10^5)^2}{10^4}\approx\boxed{9.86\times10^6\unit{m\,s^{-2}}}
$$

\question{A2}{98}{45}
\subquestion{a}
We first convert the angular speed to $\unit{Hz}$ by dividing by 60:
$$
    \omega=\frac{2}{60}=\frac{1}{30}\unit{Hz}
$$
This means the angular speed in $\unit{rad\,s^{-1}}$ is:
$$
    \omega=\frac{2\pi}{\frac{1}{30}}\approx\boxed{0.209\unit{rad\,s^{-1}}}
$$

\subquestion{b}
Since the passangers move along the circumference of the Ferris wheel we use the formula $v=\omega r$, solving for $r$:
$$
    r=\frac{v}{\omega}=\frac{3}{0.209}\approx\boxed{14.4\unit{m}}
$$

\subquestion{c}
Using $a=\frac{v^2}{r}$:
$$
    a\approx\frac{3^2}{14.4}=\boxed{0.625\unit{m\,s^{-2}}}
$$

\question{A2}{103}{46}
\subquestion{a}
Since there is no inclined plane, the normal force is equal to the weight force.
For the box to stay stationary, the frictional force has to be greater than or equal to the centripetal force:
\begin{align*}
    \force_\mu&\geq\force_C\\
    \implies mg\mu&\geq m\omega^2r
\end{align*}
Thus, solving for the maximum $\omega$ (when there is an equality rather than an inequality):
$$
    \omega\leq\sqrt{\frac{g\mu}{r}}\approx\sqrt{\frac{9.82\times0.7}{0.3}}\approx\boxed{4.79\unit{rad\,s^{-1}}}
$$

\subquestion{b}
Using $\theta=\omega t+\frac{1}{2}\alpha t^2$ (there is no acceleration in this problem), and solving for $t$:
$$
    t=\frac{\theta}{\omega}\approx\frac{2\pi}{4.79}\approx\boxed{1.31\unit{s}}
$$

\question{A2}{103}{47}
Using $a=\frac{v^2}{r}$ and solving for $r$, setting $a=9g$:
$$
    r=\frac{v^2}{a}\approx\frac{280^2}{9\times9.82}\approx\boxed{887\unit{m}}
$$

\question{A2}{103}{48}
\subquestion{a}
\graphquestion

\subquestion{b}
The vertical component of the normal force, which is $\theta$ angular units from the horizontal, must balance the weight force completley.
Thus:
\begin{align*}
    \force_N\sin\theta=\force_g&=mg\\
    \implies\force_N&=\frac{mg}{\sin\theta}
\end{align*}
The only remaining unbalanced force is then the horizontal component of the normal force, meaning by Newton's second law, the acceleration of the marble is:
\begin{align*}
    a&\newton{2}\frac{\force_N\cos\theta}{m}=\frac{\frac{mg}{\sin\theta}\cos\theta}{m}\\
    &=\frac{g\cos\theta}{\sin\theta}=\boxed{\frac{g}{\tan\theta}}
\end{align*}

\subquestion{c}
\subsubquestion{i}
Using $a=\frac{v^2}{r}$ and solving for $r$ gives (and using the expression for $a$ derived in \subquestion{a}):
\begin{align*}
    r&=\frac{v^2}{a}=\frac{v^2}{\frac{g}{\tan\theta}}\\
    &\approx\frac{1.5^2}{\frac{9.82}{\tan28^\circ}}\approx\boxed{0.122\unit{m}}
\end{align*}

\subsubquestion{ii}
What the question is really asking for is the normal force, since that is the force the bowl pushes against the marble with.
The normal force was defined in \subquestion{a}:
$$
    \force_N=\frac{mg}{\sin\theta}=\frac{0.003\times9.82}{\sin28^\circ}\approx\boxed{6.28\times10^{-2}\unit{m}}
$$

\subquestion{d}
Re-aranging $a=\omega^2r$ for $\omega$ yields:
$$
    \omega=\sqrt{\frac{a}{r}}
$$
Since $a=\frac{g}{\tan\theta}$ is constant:
$$
    \omega\propto\frac{1}{\sqrt{r}}
$$
Where $r$ is the radius of the path.
We therefore conclude that as the balls slips down, and $r$ starts decreasing, $\omega$ will increase.

\question{A2}{105}{49}
\subquestion{a}
At the bottom, since at the bottom the force of tension in the string has to exactly counter-act the weight force of the stone and the centripetal force.

\subquestion{b}
\subsubquestion{i}
Taut means not slacked.
We thus want the centripetal force to perfectly balance the tension and weight force.
At the top there is no tension due to weight force, so the equation reduces to:
\begin{align*}
    \force_g&=\force_C\\
    \implies mg&=m\frac{v^2}{r}\\
    \therefore v&=\sqrt{gr}=\sqrt{9.82\times0.8}=\boxed{2.80\unit{m\,s^{-1}}}
\end{align*}

\subsubquestion{ii}
The maximum tension (at the bottom) is given by:
$$
    \max\force_t=\force_g+\force_C=mg+m\frac{v^2}{r}
$$
We know the maximum tension force was 10$\unit{N}$, thus:
\begin{align*}
    10&\approx0.25\times9.82+0.25\times\frac{v^2}{0.8}\\
    \implies v&\approx\sqrt{\frac{10-2.455}{0.3125}}\approx\boxed{4.91\unit{m\,s^{-1}}}
\end{align*}

\question{A2}{105}{50}
\subquestion{a}
This is a tricky question.
If a question involves circular motion, then remember that the centripetal force $\force_C$ is the resultant of the forces acting on the object.
Note that the centripetal force acts downwards in towards the centre of the bridge.
The two other forces acting on the car at the top of the bridge is the weight force $\force_g$ (downward) and the normal force $\force_N$ (upward).
Thus, letting downwards towards the bridge be the direction of positive motion:
\begin{align*}
    \force_C&=\force_g-\force_N\\
    \implies m\frac{v^2}{r}&=mg-\force_N\\
    \therefore\force_N&=mg-m\frac{v^2}{r}=\boxed{m\left(g-\frac{v^2}{r}\right)}
\end{align*}

\subquestion{b}
\subsubquestion{i}
First converting 50$\unit{km\,h^{-1}}$ to $\unit{m\,s^{-1}}$:
$$
    v=\frac{50}{3.6}\approx13.9\unit{m\,s^{-1}}
$$
Using the formula derived in \subquestion{a}:
\begin{align*}
    \force_N&\approx1400\left(9.82-\frac{13.9^2}{60}\right)\\
    &\approx\boxed{9240\unit{N}}
\end{align*}

\subsubquestion{ii}
On a horizontal road, $\force_N=\force_g$, therefore, 10$\%$ of the value must be:
$$
    0.1\force_N=0.1mg\approx1370\unit{N}
$$
Then using the equation derived in \subquestion{a}:
\begin{align*}
    1370&=1400\left(g-\frac{v^2}{60}\right)\\
    \implies v&\approx\sqrt{60\left(9.82-\frac{1370}{1400}\right)}\approx\boxed{23.0\unit{m\,s^{-1}}}
\end{align*}
The answer sheet gives the answer in $\unit{km\,h^{-1}}$ which is found by taking the answer $v$ found here multiplied by 3.6.

\subsection{Subtopic A.3 – Work, energy and power}
\question{A3}{111}{1}
\subquestion{a}
Using the definition of work:
$$
    W=\force s\cos\theta=0.6\times2.5\times\cos60^\circ=\boxed{0.75\unit{J}}
$$

\subquestion{b}
Vide \subquestion{a} for explanation
$$
    W=\force s\cos\theta=0.6\times2.5\times\cos90^\circ=\boxed{0\unit{J}}
$$

\subquestion{c}
Vide \subquestion{a} for explanation
$$
    W=\force s\cos\theta=0.6\times2.5\times\cos160^\circ\approx\boxed{-1.41\unit{J}}
$$

\question{A3}{111}{2}
\subquestion{a}
Converting the speed to $\unit{m\,s^{-1}}$ by dividing by 3.6:
$$
    v=\frac{50}{3.6}\approx13.9\unit{m\,s^{-1}}
$$
Thus, the distance travelled by the car in one minute is (using $s=ut+\frac{1}{2}at^2$, with $a=0$ since the speed is constant):
$$
    s=13.9\times60\approx\boxed{833\unit{m}}
$$

\subquestion{b}
Since the speed is constant, the resultant force of the driving and resistive force must be 0.
Thus the magnitude of the two are equal.
The magnitude of the resistive force is thus (using the definition of work $W=\force s\cos\theta$, solving for $\force$ and noting that $\theta=90^\circ$):
$$
    \force\approx\frac{190000}{833}\approx\boxed{228\unit{N}}
$$

\subquestion{c}
As described in \subquestion{b}, the resistive force is equal in magnitude, but oposite in direction to the driving force.
Thus:
$$
    W_\text{resistive}=-W_\text{driving}=\boxed{-190\unit{kJ}}
$$

\question{A3}{113}{3}
\subquestion{a}
The work is the area under the curve, meaning:
$$
    W=60\times1.6\times10^3+\frac{40\times1.6\times10^3}{2}=\boxed{1.28\times10^5\unit{J}}
$$

\subquestion{b}
\subsubquestion{i}
The driving force at $s=50\unit{m}$ is $1.6\times10^3\unit{N}$, thus taking the resultant (noting the resistive force is opposite in direction) and using Newton's second law, we see:
$$
    a\newton{2}\frac{\force}{m}=\frac{1.6\times10^3-400}{1600}=\boxed{0.75\unit{m\,s^{-2}}}
$$

\subsubquestion{ii}
The driving force at $s=100\unit{m}$ is 0.
Using the same method as in \subsubquestion{i}:
$$
    a\newton{2}\frac{\force}{m}=\frac{0-400}{1600}=\boxed{-0.25\unit{m\,s^{-2}}}
$$

\question{A3}{113}{4}
\subquestion{a}
The work done is the area under the curve (under the line being negative).
Thus:
\begin{align*}
    W&=\frac{1}{2}\times3\times6-\frac{1}{2}\times2\times4-5\times4\\
    &=9-4-20=\boxed{-15\unit{J}}
\end{align*}

\subquestion{b}
The graphical way to solve this problem is noting that we need the area above and below the curve to cancel out.
The triangle above the curve is almost cancelled out by the one under, including the rectangle from $s=5$ to $6\unit{m}$.
The only area remaining is one unit triangle, which we can get from half a unit square.
By traveling an additional $0.25\unit{m}$ from $s=6\unit{m}$, we cover an area of two unit squares divided by 4, equal to half a unit square.
Thus by $s=\boxed{6.25\unit{m}}$, we have net zero work done by the force.

\question{A3}{115}{5}
Using $P=\force v\cos\theta$, solving for $\force$:
\begin{align*}
    \force=\frac{P}{v\cos\theta}&=\frac{180}{1.5\cos60^\circ}\\
    &=240\unit{N}
\end{align*}
Thus, the answer is \answer{D}.

\question{A3}{115}{6}
Using the definition of power:
\begin{align*}
    P&=\frac{W}{t}=\frac{\force s\cos\theta}{t}\\
    \leadsto P&=\force v\newton{2}mav
\end{align*}
Thus if the power is constant, then as $v$ increases, $a$ has to decrease, starting at a maximum when $v$ is low.
Then as the locomotive reaches it's "terminal" velocity, the acceleration of the locomotive starts approaching 0.
The answer is therefore \answer{D}.
The reason it can't be \textrm{A} is that since the train is driven from rest, it has to gain velocity.
Note that the power can still be constant even once the train stops accelerating, since the engine still needs to do work to opose frictional, air resistance, etc. forces.

\question{A3}{115}{7}
\subquestion{a}
The resultant force is, using Newton's second law:
$$
    \force_R\newton{2}2.5\times1600=4000\unit{N}
$$
Thus, since the only forces acting on the force is the ressistive force and the driving force:
\begin{align*}
    \force_R=4000=\force_\text{driving}-\force_\text{resistance}&=\force_\text{driving}-500\\
    \implies\force_\text{driving}&=4000+500=\boxed{4500\unit{N}}
\end{align*}

\subquestion{b}
\subsubquestion{i}
To derive the distance travelled, we use the SUVAT equation $s=ut+\frac{1}{2}at^2$:
$$
    s=0+\frac{1}{2}\times2.5\times10^2=125\unit{m}
$$
Thus, by the definition of work:
$$
    W\ipsofacto\force_\text{driving}s\cos\theta=4500\times125\approx\boxed{5.63\times10^5\unit{J}}
$$

\subsubquestion{ii}
By the definition of power, using the work found in \subquestion{a}:
$$
    P\ipsofacto\frac{W}{t}\approx\frac{5.63\times10^5}{10}=\boxed{5.63\times10^4\unit{W}}
$$

\subquestion{c}
Since a constant force opposes the motion of the car, to maintain a constant resultant acceleration, the driving force has to be constant too.
However, if the car is accelerating, the velocity is increasing.
By the definition of power:
$$
    P\ipsofacto\frac{W}{t}=\frac{\force s\cos\theta}{t}\stackrel{\text{reduces to}}{=}\force v
$$
Thus, if the force is constant:
$$
    P\propto v
$$
This means that the car must increase power to maintain a constant acceleration.

\question{A3}{116}{8}
\subquestion{a}
The work done is the area under the curve.
Thus:
$$
    W=\frac{800\times100}{2}+1200\times100=\boxed{1.6\times10^5\unit{J}}
$$

\subquestion{b}
Using the definition of work, we derive the magnitude of the force acting on the car as:
$$
    \force\ipsofacto\frac{W}{s}=\frac{1.6\times10^5}{100}={1.6\times10^3\unit{N}}
$$
Thus the acceleration is, using Newton's second law:
$$
    a\newton{2}\frac{\force}{m}=\frac{1.6\times10^3}{2500}=0.64\unit{m\,s^{-2}}
$$
Finally, using $v^2=u^2+2as$:
$$
    v=\sqrt{8^2+2\times0.64\times100}\approx\boxed{13.9 \unit{m\,s^{-1}}}
$$

\question{A3}{116}{9}
\subquestion{a}
The work done is the energy needed to halve the kinetic energy.
The original kinetic energy is thus:
$$
    E_k=\frac{1}{2}mv^2=\frac{1}{2}\times1.2\times2.7^2\approx4.37\unit{J}
$$
Thus the work done is:
$$
    W=-\frac{1}{2}E_k\approx-\frac{1}{2}\times4.37=\boxed{-2.19\unit{J}}
$$

\subquestion{b}
If the kinetic energy is halved, and the mass stays the same, then:
\begin{align*}
    \frac{1}{2}mv'^2&=\frac{1}{2}\qty(\frac{1}{2}mv^2)\\
    \implies v'^2&=\frac{v^2}{2}\\
    \therefore v'&=\sqrt{\frac{v^2}{2}}=\sqrt{\frac{2.7^2}{2}}\approx\boxed{1.91\unit{m\,s^{-1}}}
\end{align*}

\subquestion{c}
Using $s=\frac{u+v}{2}\times t$:
$$
    s\approx\frac{1.91+2.7}{2}\times0.9\approx\boxed{2.07\unit{m}}
$$

\question{A3}{120}{10}
We first need the angle between the gravitational force and the motion of the box.
Since the gravitational force is along the negative vertical, the angle is:
$$
    \theta=30+90=120^\circ
$$
Then, the gravitational force is computed using Newton's second law as:
$$
    \force_g\newton{2}mg=1.5g
$$
Finally, using the definition of work:
\begin{align*}
    W\ipsofacto\force s\cos\theta&=1.5g\times4\times\cos120^\circ\\
    &=-3g\approx30\unit{J}
\end{align*}
Thus, the answer is \answer{C}.

\question{A3}{120}{11}
The potential is given as:
$$
    E_p=mgh
$$
Note however that the initial height of the pendulum bob above it's rest position is exactly one length of the radius, meaning:
$$
    E_p=mgr
$$
Then, at the bottom, all of the potential has been converted in to kinetic energy meaning:
\begin{align*}
    E_k=E_p\implies\frac{1}{2}mv^2&=mgr\\
    \therefore v^2&=2gr
\end{align*}
Since the bob is traveling in circular motion, it experiences centripetal forces equal to the resultant force $\force_R$.
Thus:
$$
    \force_C=\frac{mv^2}{r}
$$
Using the form of $v^2$ derived earlier, we find:
$$
    \force_C=\frac{2mgr}{r}=2mg
$$
Thus, recalling that the centripetal force is the resultant, the acceleration of the bob is given by Newton's second law as:
$$
    a\newton{2}\frac{\force_R}{m}=\frac{2mg}{m}=2g
$$
Consequently, the answer is \answer{C}.

\question{A3}{121}{12}
\subquestion{a}
\subsubquestion{i}
By the definition of work:
$$
    W\ipsofacto\force s\cos\theta=3.2\times10^4\times45\approx\boxed{1.44\times10^6\unit{J}}
$$

\subsubquestion{ii}
The helicopter originally had no potential, so the change in energy is equal to the new potential which is given as:
$$
    E_p\ipsofacto mgh\approx2900\times45\times9.82\approx\boxed{1.28\times10^6\unit{J}}
$$

\subquestion{b}
The helicopter not only gained potential, but also kinetic energy.
The difference between the work done and the potential gained is the new kinetic energy of the helicopter (since it accelerated from rest).

\subquestion{c}
Using the reasoning from \subquestion{b}:
\begin{align*}
    W-E_p\approx1.44\times10^6-1.28\times10^6&\approx E_k\ipsofacto\frac{1}{2}mv^2\\
    \implies v&\approx\sqrt{\frac{2\times\qty(1.44\times10^6-1.28\times10^6)}{2900}}\approx\boxed{10.5\unit{m\,s^{-1}}}
\end{align*}

\question{A3}{121}{13}
Be careful here not to think that the entire kinetic energy of the pellet becomes converted to the kinetic energy of the block when they move together.
We instead apply the conservation of momentum.
The mass of the two objects intertwined is:
$$
    m_\text{combined}=0.25+0.0018=0.2518\unit{kg}
$$
Thus:
\begin{align*}
    m_\text{pellet}\times v_\text{pellet}&=m_\text{combined}\times v\\
    \implies v&=\frac{0.0018\times200}{0.2518}\approx1.43\unit{m\,s^{-1}}
\end{align*}
The kinetic energy of the combined body is then:
$$
    E_k\ipsofacto\frac{1}{2}mv^2=\frac{1}{2}\times0.2518\times1.43^2=0.257\unit{J}
$$
The maximum height reached is then given by converting all the kinetic energy to potential, meaning:
\begin{align*}
    0.257=E_p&\ipsofacto mgh\\
    \implies h&=\frac{0.257}{mg}\approx\frac{0.257}{0.2518\times9.82}\approx\boxed{0.104\unit{m}}
\end{align*}

\question{A3}{121}{14}
\subquestion{a}
The question asks for speed (energy), not time.
The ball has the same amount of initial kinetic energy however the ball is oriented, and gains the same amount from the initial potential.
Thus when it hits the ground it has the same kinetic energy, which means identical speeds regardless of the initial angle.

\subquestion{b}
The initial kinetic energy was:
$$
    {E_k}_\text{initial}=\frac{1}{2}mv^2=\frac{1}{2}\times m\times20^2=200m
$$
The energy gained for the final kinetic energy is the potential possesed in the beginning.
The initial potential was:
$$
    E_p=mgh=2.5mg
$$
Thus the final kinetic energy is
\begin{align*}
    {E_k}_\text{final}&={E_k}_\text{initial}+E_p\\
    &=200m+2.5mg=m(200+2.5g)
\end{align*}
Thus we compute the final velocity using the definition of kinetic energy as:
\begin{align*}
    {E_k}_\text{final}\ipsofacto\frac{1}{2}mv^2&=m(200+2.5g)\\
    \implies v&=\sqrt{400+5.g}\approx\boxed{21.2\unit{m\,s^{-1}}}
\end{align*}

\question{A3}{123}{15}
The change in elastic energy is given by:
\begin{align*}
    \Delta E_H&={E_H}_\text{final}-{E_H}_\text{initial}\\
    &=\frac{1}{2}k(2\Delta x)^2-\frac{1}{2}k\Delta x\\
    &=\frac{1}{2}k(4\Delta x-\Delta x)=3\times\frac{1}{2}k\Delta x=3{E_H}_\text{initial}
\end{align*}
Consequently, the answer is \answer{C}.

\question{A3}{123}{16}
When the block has returned to its uncompressed length, the elastic energy has been converted in to kinetic energy.
Thus:
\begin{align*}
    E_k\ipsofacto\frac{1}{2}mv^2&=E_H=\frac{1}{2}k(\Delta x)^2\\
    \implies v&=\sqrt{\frac{k(\Delta x^2)}{m}}
\end{align*}
Then note how the force exerted on the block is due to the spring force, meaning by Hooke's law:
$$
    F_H\ipsofacto-k\Delta x=k\Delta x
$$
Therefore, we can substitute in $F_H$ to the expression for the velocity:
$$
    v=\sqrt{\frac{F\Delta x}{m}}
$$
This makes the answer \answer{B}.

\question{A3}{123}{17}
\subquestion{a}
\subsubquestion{i}
If the spring is balanced, then the force used to hold it still is equivalent to the spring force, by Newton's third law.
Thus:
\begin{align*}
    60=\force_H&\ipsofacto-k\Delta x\\
    \implies k&=\frac{60}{0.03}=\boxed{2000\unit{N\,m^{-1}}}
\end{align*}

\subsubquestion{ii}
Using the definition of elastic potential as $E_H\ipsofacto\frac{1}{2}k(\Delta x)^2$:
$$
    E_H=\frac{1}{2}\times2000\times0.03^2=\boxed{0.9\unit{J}}
$$

\subquestion{b}
The workd done is converted to elastic potential, meaning the new elastic potential is given by:
$$
    E_H'=E_H+0.7=0.9+0.7=1.6\unit{J}
$$
Therefore, by the definition of elastic potential as $E_H\ipsofacto\frac{1}{2}k(\Delta x)^2$:
\begin{align*}
    1.6=E_H'&\ipsofacto\frac{1}{2}k(\Delta x)^2\\
    \implies\Delta x&=\sqrt{\frac{3.2}{2000}}=0.04\unit{m}=\boxed{4\unit{cm}}
\end{align*}

\question{A3}{123}{18}
\subquestion{a}
\subsubquestion{i}
Using the definition of kinetic energy as $E_k\ipsofacto\frac{1}{2}mv^2$:
$$
    E_k=\frac{1}{2}\times0.6\times2^2=\boxed{1.2\unit{J}}
$$

\subsubquestion{ii}
The gravitational force of the block is given by Newton's second law as:
$$
    \force_g\ipsofacto mg\approx0.6\times9.82
$$
Then, by the definition of work as $W\ipsofacto\force s\cos\theta$:
$$
    W=\force_g\times0.051\approx0.6\times9.82\times0.051\approx\boxed{0.300\unit{J}}
$$

\subsubquestion{iii}
The initial kinetic energy $E_k$ of the block has been converted to elastic potential.
However, the block has also gained energy from the block losing potential, this is equivalent to the work done by the force of gravity.
Thus:
\begin{align*}
    E_k&\rightarrow E_H+E_p\\
    \implies E_H&=E_k-E_p\approx1.2-0.3=\boxed{0.9\unit{J}}
\end{align*}

\subquestion{b}
Using $E_H=\frac{1}{2}k(\Delta x)^2$, using the result from \subquestion{a}:
\begin{align*}
    0.9&\approx\frac{1}{2}k(0.051)^2\\
    \implies k&=\frac{2\times0.9}{0.051^2}\approx\boxed{692\unit{N\,m^{-1}}}
\end{align*}

\subquestion{c}
The forces acting on the block are the force of gravity and the spring force (by Hooke's law) $\force_H=-k\Delta x$.
Thus, the resultant is:
$$
    \force_R=\force_H-\force_g\newton{2}692\times0.051-0.6\times9.82=29.4\unit{N}
$$
Therefore, by Newton's second law:
$$
    a\newton{2}\frac{\force_R}{m}\approx\frac{29.4}{0.6}=\boxed{49\unit{m\,s^{-2}}}
$$

\question{A3}{125}{19}
Since $P=\frac{W}{t}$:
$$
    W=Pt=270\times200=54000\unit{J}
$$
The potential energy gained by the cyclist is given as:
$$
    \Delta E_p=mg\Delta h=85\times g\times50=41700\unit{J}
$$
Thus, the efficiency is:
$$
    \eta=\frac{\Delta E_p}{W}\approx\frac{41700}{54000}\approx\boxed{77.3\%}
$$

\question{A3}{125}{20}
Converting the speed of $50\unit{km\,h^{-1}}$ to $\unit{m\,s^{-1}}$:
$$
    v=\frac{50}{3.6}=13.\bar{8}\unit{m\,s^{-1}}
$$
The kinetic energy gained (since there was none at the start, when the car was at rest) is:
$$
    E_k=\frac{1}{2}mv^2=\frac{1}{2}\times1600\times13.\bar{8}^2\approx154000\unit{J}
$$
Thus, if this is only 65\% of the energy transffered, then the total energy transfered is:
$$
    E=\frac{E_k}{65\%}\approx\frac{154000}{0.65}\approx\boxed{2.37\times10^5\unit{J}}
$$

\question{A3}{128}{21}
\subquestion{a}
Since the energy losses listed are all the energy losses, $100\%-\text{losses}=\eta$, where $\eta$ is the efficiency.
Thus:
\begin{align*}
    \eta&=100\%-27\%-15\%-5\%\\
    &=1-0.27-0.15-0.05=0.53=\boxed{53\%}
\end{align*}

\subquestion{b}
\graphquestion

\subsection{Subtopic A.4 – Rigid body mechanics}
\question{A4}{133}{1}
\subquestion{a}
Using $\omega_f=\omega_i+\alpha t$:
$$
    \alpha=\frac{\omega_f-\omega_i}{t}=\frac{15-5}{20}=\boxed{0.5\unit{rad\,s^{-2}}}
$$

\subquestion{b}
Using $\omega_f^2=\omega_i^2+2\alpha\theta$, solving for $\theta$:
\begin{align*}
    \theta&=\frac{\omega_f^2-\omega_i^2}{2\alpha}=\frac{15^2-5^2}{2\times0.5}\\
    &=200\unit{rad}
\end{align*}
Therefore, the amount of revolutions is:
$$
    \frac{200}{2\pi}\approx\boxed{31.8\unit{revolutions}}
$$

\question{A4}{133}{2}
\subquestion{a}
Using $\omega_f=\omega_i+\alpha t$:
$$
    \omega_f=0+4.7\times20=94\unit{rad\,s^{-1}}
$$
Multiplying by 60 (for the amount per minute) and dividing by $2\pi$ (for the amount in revolutions), gives:
$$
    \frac{94\times60}{2\pi}=\boxed{898\unit{revolutions\,minute^{-1}}}
$$

\subquestion{b}
Using $\omega_f^2=\omega_i^2+2\alpha\theta$, solving for $\theta$:
$$
    \theta=\frac{\omega_f^2-\omega_i^2}{2\alpha}=\frac{94^2-0}{2\times4.7}=940\unit{rad}
$$
Thus, in revolutions:
$$
    \frac{940}{2\pi}\approx\boxed{150\unit{revolutions}}
$$

\question{A4}{133}{3}
\subquestion{a}
Converting the speed of 1800$\unit{revolutions\,minute^{-1}}$ to $\unit{rad\,s^{-1}}$:
$$
    \omega=\frac{1800\times2\pi}{60}=188\unit{rad\,s^{-1}}
$$
Using $\omega_f=\omega_i+\alpha t$, solving for $\alpha$:
$$
    \alpha=\frac{\omega_f-\omega_i}{t}=\frac{188-0}{0.1}=\boxed{1880\unit{rad\,s^{-2}}}
$$

\subquestion{b}
Using $\displaystyle\theta=\qty(\frac{\omega_f+\omega_i}{2})t$, solving for $t$:
$$
    t=\frac{2\theta}{\omega_f+\omega_i}\approx\frac{2\times5\times2\pi}{0+188}\approx\boxed{0.334\unit{s}}
$$

\question{A4}{133}{4}
\subquestion{a}
Using $\omega_f^2=\omega_i^2+2\alpha\theta$, solving for $\alpha$:
$$
    \alpha=\frac{\omega_f^2-\omega_i^2}{2\theta}=\frac{50^2-10^2}{2\times20\times2\pi}\approx\boxed{9.55\unit{rad\,s^{-1}}}
$$

\subquestion{b}
\subsubquestion{i}
Using $\displaystyle\theta=\qty(\frac{\omega_f+\omega_i}{2})t$, solving for $t$:
$$
    t=\frac{2\theta}{\omega_f+\omega_i}\approx\frac{2\times20\times2\pi}{50+10}\approx\boxed{4.19\unit{s}}
$$

\subsubquestion{ii}
Using $\omega_f^2=\omega_i^2+2\alpha\theta$, solving for $\omega_f$:
$$
    \omega_f\approx\sqrt{10+2\times9.55\times10\times2\pi}\approx34.8\unit{rad\,s^{-1}}
$$
Thus, by $\omega_f=\omega_i+\alpha t$, solving for $t$:
$$
    t=\frac{\omega_f-\omega_i}{\alpha}\approx\frac{34.8-10}{9.55}\approx\boxed{2.60\unit{s}}
$$

\question{A4}{137}{5}
\subquestion{a}
If the balls have equal mass, then, since ball A has a lower density, it needs a greater volume to achieve the same mass.
To achieve the greater volume, it needs a larger radius.
By the definition of the moment of inertia for a sphere:
$$
    I=mr^2
$$
Thus, if the mass is the same, then \answer{ball A} must be the ball of greater moment of inertia.

\subquestion{b}
If the radius is the same, then the balls have equal volume and the $r^2$ term in the moment of inertia does not matter.
Thus, since \answer{ball B} has greater density, it will have more mass, meaning it will have the greater moment of inertia.

\question{A4}{137}{6}
Do not forget that the mass, too, changes due to the change in radius.
The change in the volume $\displaystyle V=\frac{4}{3}\times\pi\times r^3$, is proportional as:
$$
    V\propto r^3
$$
The change in moment of inertia $I=mr^2$, is proportional as:
$$
    I=mr^2=V\rho r^2\propto r^5
$$
Thus, if the radius is halved:
$$
    I_\text{radius halved}=I\times\qty(\frac{1}{2})^5=\frac{I}{32}
$$
Consequently, the answer is \answer{A}.

\question{A4}{139}{7}
\subquestion{a}
For the rod to remain in equilibrium, the torques have to balance.
Thus, using $\tau=\force r\sin\theta$, with $\force\newton{2}mg$:
\begin{align*}
    \tau_1&=\tau_2\\
    \implies1.5g\times0.8&=mg\times1.0\\
    \therefore m&=1.5\times0.8=\boxed{1.2\unit{kg}}
\end{align*}

\subquestion{b}
The normal force is the correspondent to the weight force.
The total weight of the rod is (using the result from \subquestion{a}):
$$
    m=1.5+1.2=2.7\unit{kg}
$$
Thus, the normal force is:
$$
    \force_N=\force_g=2.7g\approx\boxed{26.5\unit{N}}
$$

\question{A4}{139}{8}
\subquestion{a}
\graphquestion

\subquestion{b}
First, the torques in the problem have to be idenetified.
The friction at the bottom contact of the ladder is making it a stationary point.
Therefore, the torque acting from gravity (half way along $r$, at a $35^\circ$ angle) must balance the torque which is due to the wall pushing away on the rod through normal force ${\force_N}_\text{wall}$.
Thus:
\begin{align*}
    \tau_g\newton{2}12g\frac{r}{2}\sin35^\circ&=\tau_N={\force_N}_\text{wall}r\sin55^\circ\\
    \implies{\force_N}_\text{wall}&=\frac{6g\sin35^\circ}{\sin55^\circ}\approx\boxed{41.3\unit{N}}
\end{align*}

\subquestion{c}
Since the normal force from part \subquestion{b} is the only other horizontal force, $\force_\mu={\force_N}_\text{wall}$.
The force of friction here comes from the vertical normal force, which is equal to the weight force:
$$
    {\force_N}_\text{ground}=\force_g=mg\approx12\times9.82
$$
Thus:
\begin{align*}
    {\force_N}_\text{ground}\mu_s&={\force_N}_\text{wall}\\
    \implies\mu_s&=\frac{{\force_N}_\text{wall}}{{\force_N}_\text{ground}}\approx\frac{41.3}{12\times9.82}\approx\boxed{0.350}
\end{align*}

\question{A4}{139}{9}
\subquestion{a}
The torque from the weight force (half way along $r$, at a 30$^\circ$ angle) must balance the torque from the tension.
Therefore:
\begin{align*}
    \tau_g\newton{2}40g\times\frac{r}{2}\times\sin30^\circ&=\tau_T=\force_T\times r\times\sin90^\circ\\
    \implies\force_T&=20g\times\sin30^\circ=10g\approx\boxed{98.2\unit{N}}
\end{align*}

\subquestion{b}
The two/three forces acting vertically is: the normal force (the one in question), the force of gravity and the vertical component of the force of tension.
Because we see that the vertical part of the force of tension $\force_T\sin30^\circ$ is less than the force of gravity $40g$, we the vertical component of the normal contact force must be non zero to keep translational equilibrium.

\question{A4}{139}{10}
Note that the parts of the rod support a proportionally equal part of the mass which is given by:
$$
    m'=\frac{D-d}{D}\times m
$$
where $D$ is the total length, and $d$ the length of the segment.
Thus, the mass experienced at $P$ is:
$$
    m_P=\frac{0.5-0.1}{0.5}\times m=0.8m
$$
Analagously, the force of gravity experienced must then be:
$$
    \force_g\newton{2}0.8mg
$$
Therefore, the answer is \answer{D}.

\question{A4}{142}{11}
The torque, being a couple, is given simply as:
$$
    \tau=\force L
$$
The moment of inertia, having 2 equidistant points, is given as:
$$
    I=2m\qty(\frac{L}{2})^2=\frac{mL^2}{2}
$$
Thus, by Newton's second law:
$$
    \alpha\newton{2}\frac{\tau}{I}=\frac{\force L}{\frac{mL^2}{2}}=\frac{2\force}{mL}
$$
Consequently, the answer is \answer{B}.

\question{A4}{142}{12}
\subquestion{a}
The rod is modelled as a long disc, which means that the moment of inertia is given by:
$$
    I=\frac{1}{2}mR^2=\frac{1}{2}\times1.2\times0.04^2=0.6\times0.04^2
$$
The torque given from the force is then given using $\tau=\force r\sin\theta$:
$$
    \tau=1.8\times0.04
$$
Thus, by Newton's second law:
$$
    \alpha\newton{2}\frac{\tau}{I}=\frac{1.8\times0.04}{0.6\times0.04^2}=3\times\frac{1}{0.04}=\boxed{75\unit{rad\,s^{-2}}}
$$

\subquestion{b}
The string is wrapped five times which means that the degrees needed to turn around is:
$$
    \theta=5\times2\pi=10\pi
$$
Thus, using $\omega_f^2=\omega_i^2+2\alpha\theta$:
$$
    \omega_f=\sqrt{0+2\times75\times10\pi}=\sqrt{1500\pi}\approx\boxed{68.6\unit{rad\,s^{-1}}}
$$
Using $\displaystyle\theta=\frac{\omega_i+\omega_f}{2}\times t$:
\begin{align*}
    10\pi&\approx\frac{68.6+0}{2}\times t\\
    \implies t&\approx\frac{20\pi}{68.6}\approx\boxed{0.916\unit{s}}
\end{align*}

\question{A4}{142}{13}
\subquestion{a}
Using $\omega_f^2=\omega_i^2+2\alpha\theta$:
\begin{align*}
    0&=23^2+2\alpha\times\frac{2\pi}{4}\\
    \leadsto\alpha&=\frac{4\times23^2}{4\pi}=\frac{23^2}{\pi}\approx\boxed{168\unit{rad\,s^{-2}}}
\end{align*}
Then, by Newton's second law:
$$
    \tau\newton{2}I\alpha\approx0.09\times168\approx\boxed{15.1\unit{N\,m}}
$$

\subquestion{b}
Since the torque acting is frictional, we know that using $\tau=\force r\sin\theta$, and since there are two of these frictional forces contributing to the torque:
$$
    \tau=\force_\mu\times0.08=2\force_N\times\mu_d\times0.08
$$
Thus, using the values computed in \subquestion{a}:
\begin{align*}
    15.1&=2\force_N\times0.85\times0.08\\
    \implies\force_N&=\frac{15.1}{2\times0.85\times0.08}\approx\boxed{111\unit{N}}
\end{align*}

\question{A4}{142}{14}
\subquestion{a}
Converting the inital angular velocity to $\unit{rad\,s^{-1}}$:
$$
    \omega=\frac{320\times2\pi}{60}=\frac{64\pi}{6}
$$
Using $\omega_f=\omega_i+\alpha t$:
\begin{align*}
    0&=\omega_i+\alpha t\\
    \leadsto\alpha&=\frac{\omega_i}{t}=\frac{\frac{64\pi}{6}}{8}\approx4.19\unit{rad\,s^{-2}}
\end{align*}
Then using Newton's second law:
\begin{align*}
    \tau&\newton{2}I\alpha\\
    \implies I&=\frac{\tau}{\alpha}=\frac{0.1}{4.19}\approx\boxed{0.0239\unit{kg\,m^2}}
\end{align*}

\subquestion{b}
Using $\displaystyle\theta=\frac{\omega_i+\omega_f}{2}\times t$:
$$
    \theta=\frac{\frac{64\pi}{6}+0}{2}\times8=\frac{32\pi}{6}\times8\approx134\unit{rad}\approx\boxed{21.3\unit{revolutions}}
$$

\question{A4}{147}{15}
\subquestion{a}
Using the formula given:
$$
    I=\frac{2}{3}mR^2=\frac{2}{3}\times0.45\times0.11^2=0.00363\unit{kg\,m^2}
$$

\end{A}