\backgroundsetup
{
    scale=1,
    angle=0,
    opacity=1,
    contents={\begin{tikzpicture}[remember picture,overlay]
        \path [fill=Space] (-\paperwidth, 0.375\paperheight)rectangle (\paperwidth, \paperheight); 
    \end{tikzpicture}}
}
\BgThispage

\begin{A}
\section{Topic A – Space, time and motion}
\subsection{Subtopic A.1 – Kinematics}
\question{A1}{11}{1}
\subquestion{a}
The total distance is given as the sum of distance traveled during each translation, $2.5+3.8=6.3\unit{km}$

\subquestion{b}
Displacement is a vector quantity, thus taking the magnitude of the sum of the two vector displacements caused by the movement:
\begin{align*}
    \magnitude{\vector{2.5}{0}+\vector{0}{3.8}}&=\magnitude{\vector{2.5}{3.8}}\\
    &=\sqrt{2.5^2+3.8^2}\approx\boxed{4.55\unit{km}}
\end{align*}

\subquestion{c}
Since this can be scenario can be set up as a right angle triangle, with the boat as the hypothesis' outer vertex, using trigenometry the angle can is determined as:
$$
    \tan\theta=\frac{o}{a}\leadsto\theta=\tan^{-1}\frac{2.5}{3.8}\approx33.3^\circ
$$

\question{A1}{11}{2}
\subquestion{a}
15 minutes corresponds to a 90 degree ($\frac{\pi}{2}\unit{rad}$) rotation on a clock, or one quarter of its perimeter.
Therefore the distance travelled by the tip of the pointer must be:
$$
    s=\frac{2\pi r}{4}=\frac{15\pi}{2}\approx23.6\unit{cm}
$$
The displacement however is the distance between the points $\point{0}{15}$ and $\point{15}{0}$, thus using Pythagoras' theorem\footnote{$a^2+b^2=c^2$}:
$$
    s=\sqrt{15^2+15^2}=\sqrt{450}=21.2\unit{cm}
$$

\subquestion{b}
Analagously to question a but for the 180 degrees ($\pi\unit{rad}$) rotation resulting from 30 elapsed minutes:
\begin{align*}
    &s_{\text{distance}}=\frac{2\pi r}{2}=15\pi\approx47.1\unit{cm}\\
    &s_{\text{displacement}}=\sqrt{0^2+30^2}=30\unit{cm}
\end{align*}

\question{A1}{12}{3}
Since they are headed in completley oposite directions, $s_\text{Ada}+s_\text{Matt}=580\unit{m}$.
Ada's speed of $20\unit{km\,h^{-1}}$ is approximately $5.56\unit{m\,s^{-1}}$, as found by dividing through by 3.6.
Since $\displaystyle s=\int v\dd{t}$:
\begin{align*}
    v_\text{Ada}t+v_\text{Matt}t&=580\\
    5.56\times60+v_\text{Matt}\times60&=580\\
    \therefore v_\text{Matt}&\approx4.11\unit{m\,s^{-1}}\equiv\boxed{14.8\unit{km\,h^{-1}}}
\end{align*}

\question{A1}{12}{4}
First multiply the speed of light by the length of a light-year in seconds to obtain $1\unit{ly}$.
\begin{align*}
    1\unit{ly}&=3\times10^8\unit{m\,s^{-1}}\times(60\times60\times24\times365)\unit{s}\\
    &\approx9.46\times10^{15}
\end{align*}
Then, find the conversion between $1\unit{au}$ and one light-year.
$$
    1\unit{au}\approx\frac{1.5\times10^{11}}{9.46\times10^{15}}=1.59\times10^{-5}\unit{ly}
$$
Therefore,
$$
    5.5\times10^5\unit{au}\approx\boxed{8.72\unit{ly}}
$$

\question{A1}{12}{5}
\subquestion{a}
Calculating the speed between stations A and B:
$$
    v=\frac{1000}{80}=12.5\unit{m\,s^{-1}}
$$
Multiplying by $3.6$ to obtain the speed in $\unit{km\,h^{-1}}$:
$$
    v=12.5\times3.6=\boxed{45\unit{km\,h^{-1}}}
$$

\subquestion{b}
$\Delta y=1800-1000=800\unit{m}$

\subquestion{c}
The train travels $800\unit{m}$ in $60$ seconds, therefore, analagously to part \subquestion{a}:
$$
    v=\frac{800}{60}\approx13.3\unit{m\,s^{-1}}=\boxed{47.9\unit{km\,h^{-1}}}
$$
\end{A}