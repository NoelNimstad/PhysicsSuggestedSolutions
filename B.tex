\backgroundsetup
{
    scale=1,
    angle=0,
    opacity=1,
    contents={\begin{tikzpicture}[remember picture,overlay]
        \path [fill=Particulate] (-\paperwidth, 0.375\paperheight)rectangle (\paperwidth, \paperheight); 
    \end{tikzpicture}}
}
\BgThispage

\newcommand{\SpecificHeatCapacityUnit}{J\,kg^{-1}\,K^{-1}}

\begin{B}
\section{Topic B – The particulate nature of matter}
\subsection{Subtopic B.1 – Thermal energy transfers}
\question{B1}{202}{1}
\subquestion{a}
The physical state (e.g. solid, liquid, gas) of an object varies with temperature.

\subquestion{b}
For the answer from \subquestion{a} you can't directly measure the temperature, only an esimate.
For example, for water, if it's solid (ice), then we know the temperature is roughly 100$^\circ$.

\question{B1}{204}{2}
We first need to conver the celcius measurements to kelvin, which is done by:
\begin{align*}
    &T_0=100\unit{^\circ C}\approx100+273=373\unit{^\circ K}\\
    &T_1=470\unit{^\circ C}\approx470+273=743\unit{^\circ K}
\end{align*}
The formula for the energy of a gas is given by:
$$
    E=\frac{3}{2}k_bT
$$
Thus:
\begin{align*}
    &E_0=\frac{3}{2}\times373\times k_b\\
    &E_1=\frac{3}{2}\times743\times k_b=\frac{743}{373}E_0\approx2E_0
\end{align*}
The change in energy is therefore:
$$
    E_1-E_0=2E_0-E_0=E_0
$$
The answer is therefore \answer{A}.

\question{B1}{204}{3}
The gas has to be at the same temperature.
Thus the energy of both the mixtures have the same average energies, given by:
$$
    E_k=\frac{3}{2}k_bT
$$
Equating this energy to the kinetic energy gives:
\begin{align*}
    E_k=\frac{3}{2}k_bT&=E_\text{kinetic}=\frac{1}{2}mv^2\\
    \implies v=\sqrt{\frac{3k_bT}{m}}
\end{align*}
The crucial thing to note here is that:
$$
    v\propto\sqrt{\frac{1}{m}}
$$
Let the mass of the $\mathrm{O}_2$ sample be denoted $M$.
The mass of the $\mathrm{N}_2$ sample would then be:
$$
    \frac{\text{mass of }\mathrm{O}_2}{\text{mass of }\mathrm{N}_2}=\frac{8}{7}\implies\text{mass of }\mathrm{N}_2=\frac{7M}{8}
$$
Therefore, the fraction of there average velocity is given as, using the previously defined proportionality:
$$
    \frac{\sqrt{\frac{1}{M}}}{\sqrt{\frac{1}{\frac{7M}{8}}}}=\sqrt{\frac{7M}{8M}}=\sqrt{\frac{7}{8}}
$$
The answer is thus \answer{B}.

\question{B1}{209}{4}
\subquestion{a}
The molecules inside each liquid moves at different average velocities, given by their initial energy.
The molecules then collide, transferring energy to each other, until the average velocity has evened out, which would make the average energy even out.
At this point, if the average energies are equal, then the substance as a whole must have the same temperature, since:
$$
    E_k=\frac{3}{2}k_b\Delta T\implies E_k\not\propto m
$$

\subquestion{b}
Using $Q=mc\Delta t$, we find:
\begin{align*}
    &\text{(metal) }Q=0.1c_m\times(90-30)=6c_m\\
    &\text{(liquid) }Q=0.15c_m\times(30-20)=1.5c_l
\end{align*}
Thus, since the energy transfered must be equal:
$$
    6c_m=1.5c_l
$$
Dividing first by $c_m$ and then by $1.5$ gives us the ratio we want:
$$
    \frac{c_l}{c_m}=\frac{6}{1.5}=4
$$
The answer is therefore \answer{B}.

\question{B1}{209}{5}
Note that since the samples are of the same material, the specific heat capacity $c$ is the same.
Therefore, equating the energy transfered in $Q=mc\Delta T$:
\begin{align*}
    0.6c\times(30-10)&=mc\times(40-30)\\
    \implies m&=0.6\times\frac{20}{10}=1.2\unit{kg}
\end{align*}
The answer is consequently \answer{D}.

\question{B1}{209}{6}
Using $Q=mc\Delta T$:
\begin{align*}
    8300&=0.3c\times60\\
    \implies c&=\frac{8300}{0.3\times60}\approx\boxed{461\unit{\SpecificHeatCapacityUnit}}
\end{align*}

\question{B1}{209}{7}
Noting again that the energy transfered is equal, we see that (uising $Q=mc\Delta t$):
\begin{align*}
    920m\times(87-20)&=mc\times(250-87)\\
    \implies c&=\frac{920\times67}{163}\approx\boxed{378\unit{\SpecificHeatCapacityUnit}}
\end{align*}

\question{B1}{209}{8}
\subquestion{a}
Using $Q=mc\Delta t$:
\begin{align*}
    640&=0.08\times127\times\Delta T\\
    \implies\Delta T&=\frac{640}{0.08\times127}\approx63.0\unit{^\circ K}
\end{align*}
Thus, the final temperature is:
$$
    T_\text{final}=T_0+\Delta T\approx22+63=\boxed{85\unit{^\circ C}}
$$

\subquestion{b}
We use and equate $Q=mc\Delta T$ for the two samples (using the specific heat capacity of water given by the book):
\begin{align*}
    0.08\times127\times\Delta T_\text{ball}&=0.16\times4200\times\Delta T_\text{water}\\
    \implies\frac{\Delta T_\text{water}}{\Delta T_\text{ball}}&=\frac{0.08\times127}{0.16\times4200}\approx0.0151
\end{align*}

\question{B1}{209}{9}
\subquestion{a}
The total energy transfered is given by:
$$
    Q=Pt
$$
Therefore, using $Q=mc\Delta T$:
\begin{align*}
    2200t&=0.9\times4200\times(95-20)\\
    \implies t&=\frac{0.9\times4200\times75}{2200}\approx\boxed{129\unit{s}}
\end{align*}

\subquestion{b}
The extra energy supplied over what is needed is given as:
$$
    Q=2200\times150-2200\times120=2200(150-129)=2200\times21
$$
Thus, by the definition of power as $\displaystyle P=\frac{Q}{t}$ means:
$$
    P=\frac{2200\times21}{150}=\boxed{308\unit{W}}
$$

\question{B1}{209}{10}
Using $Q=mc\Delta T$ we determine the power transfered over 1 minute as:
$$
    Q=0.5\times4200\times35=73500
$$
Then, dividing by the time in one minute, we determine the power (energy transferd per unit time):
$$
    P=\frac{Q}{60}=\frac{73500}{60}\approx\boxed{1230\unit{W}}
$$

\question{B1}{213}{11}
Since water has a relatively high specific heat capacity, it absorbs more energy before heating up.
Therefore, the energy needed to heat up the environment is greater, making it heat up slower since the energy transfered to the environment is mostly constant.

\question{B1}{213}{12}
\subquestion{a}
Using $Q=mL$:
$$
    Q=0.05\times330\times1000=16500=\boxed{16.5\unit{kJ}}
$$

\subquestion{b}
The water needs to transfer as much energy as was computed in \subquestion{a} to completley melt the ice.
Therefore, using $Q=mc\Delta T$:
\begin{align*}
    16500&=4200m\times20\\
    \implies m&=\frac{16500}{4200}\times\frac{1}{20}\approx\boxed{0.196\unit{kg}}
\end{align*}

\subquestion{c}
It will be a mixture of ice and water at 0$\unit{^\circ C}$.

\question{B1}{213}{13}
\subquestion{a}
If the ice is at 0, then it only undergoes freezing (meaning we use the latent heat capacity $\displaystyle Q=mL$), meaning:
\begin{align*}
    m_\text{water}\times c_\text{water}\times(30-10)&=m\times330000\\
    \implies 0.5\times4200\times20&=330000m\\
    \implies m&=\frac{2100\times20}{330000}\approx\boxed{0.127\unit{kg}}
\end{align*}

\subquestion{b}
Now the energy transfered comes not only from the latent heat capacity, but the specific heat capacity as well.
This is because the ice first has to reach 0 degrees before it can start changing phase.
Thus we sum the energy contributions as:
\begin{align*}
    m_\text{water}\times c_\text{water}\times(30-10)&=m\times330000+m\times c_\text{water}\times(0-(-18))\\
    \implies 0.5\times4200\times20&=330000m+4200\times18\times m\\
    \therefore m&=\frac{42000}{330000+4200\times18}\approx\boxed{0.104\unit{kg}}
\end{align*}

\question{B1}{213}{14}

\subsection{Subtopic B.2 – Greenhouse effect}
\question{B2}{235}{1}
Using $P=eA\sigma T^4$:
\begin{align*}
    1100&=0.9\times1.4\times\sigma\times T^4\\
    \implies T&=\sqrt{\frac{1100}{0.9\times1.4\times5.67\times10^{-8}}}\approx\boxed{352\unit{K}}
\end{align*}

\question{B2}{235}{2}

\question{B2}{235}{3}
\subquestion{a}
Using $P=eA\sigma T^4$:
$$
    P=0.75\times0.15^2\times6\times\sigma\times(273.15)^4\approx\boxed{32.0\unit{W}}
$$

\subquestion{b}
Differentiating $Q=mc\Delta T$ with respect to $t$ for some infinitesimal $\Delta T$, noting that $\displaystyle\dv{Q}{t}=P$, gives us:
$$
    \dv{Q}{t}=P=mc\dv{T}{t}
$$
Thus, using the data given:
\begin{align*}
    \dv{T}{t}=\frac{P}{mc}\approx\frac{32.0}{28\times380}\approx0.00301=\boxed{3.01\times10^{-3}\unit{K\,s^{-1}}}
\end{align*}

\subquestion{c}
There are more ways of energy transfer than pure thermal radiation.
For example, the contact of the box with the floor will transfer heat through thermal convection.

\question{B2}{236}{5}
Using $P=IA$, noting that the average intensity on earth is $1400\unit{J}$, we get that the input power is:
$$
    P_\text{in}=1400\times16=22400\unit{W}
$$
Thus, the power output is:
$$
    P_\text{out}=\eta P_\text{in}=0.1\times22400=\boxed{2240\unit{W}}
$$

\question{B2}{240}{7}
Using $P=e\times IA$
\begin{align*}
    43000&=e\times900\times64\\
    \implies e&=\frac{43000}{900\times64}\approx0.747
\end{align*}
Thus, since $a=1-e$:
$$
    a\approx1-0.747=\boxed{0.253}
$$

\question{B2}{243}{9}
The particles can absorb the energy emitted and then they re-emit it in random directions per infinitesimal time unit.
Thus some of this energy is returned to the earth's surface.

\question{B2}{251}{11}

\subsection{Subtopic B.4 – Thermodynamics}
\question{B4}{276}{1}
Using $Q=\Delta U+W$ we get:
$$
    \Delta U=Q-W=30-50=-20\unit{J}
$$
The answer is therefore \answer{B}.

\end{B}